
% Default to the notebook output style

    


% Inherit from the specified cell style.




    
\documentclass[11pt]{article}

    
    
    \usepackage[T1]{fontenc}
    % Nicer default font (+ math font) than Computer Modern for most use cases
    \usepackage{mathpazo}

    % Basic figure setup, for now with no caption control since it's done
    % automatically by Pandoc (which extracts ![](path) syntax from Markdown).
    \usepackage{graphicx}
    % We will generate all images so they have a width \maxwidth. This means
    % that they will get their normal width if they fit onto the page, but
    % are scaled down if they would overflow the margins.
    \makeatletter
    \def\maxwidth{\ifdim\Gin@nat@width>\linewidth\linewidth
    \else\Gin@nat@width\fi}
    \makeatother
    \let\Oldincludegraphics\includegraphics
    % Set max figure width to be 80% of text width, for now hardcoded.
    \renewcommand{\includegraphics}[1]{\Oldincludegraphics[width=.8\maxwidth]{#1}}
    % Ensure that by default, figures have no caption (until we provide a
    % proper Figure object with a Caption API and a way to capture that
    % in the conversion process - todo).
    \usepackage{caption}
    \DeclareCaptionLabelFormat{nolabel}{}
    \captionsetup{labelformat=nolabel}

    \usepackage{adjustbox} % Used to constrain images to a maximum size 
    \usepackage{xcolor} % Allow colors to be defined
    \usepackage{enumerate} % Needed for markdown enumerations to work
    \usepackage{geometry} % Used to adjust the document margins
    \usepackage{amsmath} % Equations
    \usepackage{amssymb} % Equations
    \usepackage{textcomp} % defines textquotesingle
    % Hack from http://tex.stackexchange.com/a/47451/13684:
    \AtBeginDocument{%
        \def\PYZsq{\textquotesingle}% Upright quotes in Pygmentized code
    }
    \usepackage{upquote} % Upright quotes for verbatim code
    \usepackage{eurosym} % defines \euro
    \usepackage[mathletters]{ucs} % Extended unicode (utf-8) support
    \usepackage[utf8x]{inputenc} % Allow utf-8 characters in the tex document
    \usepackage{fancyvrb} % verbatim replacement that allows latex
    \usepackage{grffile} % extends the file name processing of package graphics 
                         % to support a larger range 
    % The hyperref package gives us a pdf with properly built
    % internal navigation ('pdf bookmarks' for the table of contents,
    % internal cross-reference links, web links for URLs, etc.)
    \usepackage{hyperref}
    \usepackage{longtable} % longtable support required by pandoc >1.10
    \usepackage{booktabs}  % table support for pandoc > 1.12.2
    \usepackage[inline]{enumitem} % IRkernel/repr support (it uses the enumerate* environment)
    \usepackage[normalem]{ulem} % ulem is needed to support strikethroughs (\sout)
                                % normalem makes italics be italics, not underlines
    

    
    
    % Colors for the hyperref package
    \definecolor{urlcolor}{rgb}{0,.145,.698}
    \definecolor{linkcolor}{rgb}{.71,0.21,0.01}
    \definecolor{citecolor}{rgb}{.12,.54,.11}

    % ANSI colors
    \definecolor{ansi-black}{HTML}{3E424D}
    \definecolor{ansi-black-intense}{HTML}{282C36}
    \definecolor{ansi-red}{HTML}{E75C58}
    \definecolor{ansi-red-intense}{HTML}{B22B31}
    \definecolor{ansi-green}{HTML}{00A250}
    \definecolor{ansi-green-intense}{HTML}{007427}
    \definecolor{ansi-yellow}{HTML}{DDB62B}
    \definecolor{ansi-yellow-intense}{HTML}{B27D12}
    \definecolor{ansi-blue}{HTML}{208FFB}
    \definecolor{ansi-blue-intense}{HTML}{0065CA}
    \definecolor{ansi-magenta}{HTML}{D160C4}
    \definecolor{ansi-magenta-intense}{HTML}{A03196}
    \definecolor{ansi-cyan}{HTML}{60C6C8}
    \definecolor{ansi-cyan-intense}{HTML}{258F8F}
    \definecolor{ansi-white}{HTML}{C5C1B4}
    \definecolor{ansi-white-intense}{HTML}{A1A6B2}

    % commands and environments needed by pandoc snippets
    % extracted from the output of `pandoc -s`
    \providecommand{\tightlist}{%
      \setlength{\itemsep}{0pt}\setlength{\parskip}{0pt}}
    \DefineVerbatimEnvironment{Highlighting}{Verbatim}{commandchars=\\\{\}}
    % Add ',fontsize=\small' for more characters per line
    \newenvironment{Shaded}{}{}
    \newcommand{\KeywordTok}[1]{\textcolor[rgb]{0.00,0.44,0.13}{\textbf{{#1}}}}
    \newcommand{\DataTypeTok}[1]{\textcolor[rgb]{0.56,0.13,0.00}{{#1}}}
    \newcommand{\DecValTok}[1]{\textcolor[rgb]{0.25,0.63,0.44}{{#1}}}
    \newcommand{\BaseNTok}[1]{\textcolor[rgb]{0.25,0.63,0.44}{{#1}}}
    \newcommand{\FloatTok}[1]{\textcolor[rgb]{0.25,0.63,0.44}{{#1}}}
    \newcommand{\CharTok}[1]{\textcolor[rgb]{0.25,0.44,0.63}{{#1}}}
    \newcommand{\StringTok}[1]{\textcolor[rgb]{0.25,0.44,0.63}{{#1}}}
    \newcommand{\CommentTok}[1]{\textcolor[rgb]{0.38,0.63,0.69}{\textit{{#1}}}}
    \newcommand{\OtherTok}[1]{\textcolor[rgb]{0.00,0.44,0.13}{{#1}}}
    \newcommand{\AlertTok}[1]{\textcolor[rgb]{1.00,0.00,0.00}{\textbf{{#1}}}}
    \newcommand{\FunctionTok}[1]{\textcolor[rgb]{0.02,0.16,0.49}{{#1}}}
    \newcommand{\RegionMarkerTok}[1]{{#1}}
    \newcommand{\ErrorTok}[1]{\textcolor[rgb]{1.00,0.00,0.00}{\textbf{{#1}}}}
    \newcommand{\NormalTok}[1]{{#1}}
    
    % Additional commands for more recent versions of Pandoc
    \newcommand{\ConstantTok}[1]{\textcolor[rgb]{0.53,0.00,0.00}{{#1}}}
    \newcommand{\SpecialCharTok}[1]{\textcolor[rgb]{0.25,0.44,0.63}{{#1}}}
    \newcommand{\VerbatimStringTok}[1]{\textcolor[rgb]{0.25,0.44,0.63}{{#1}}}
    \newcommand{\SpecialStringTok}[1]{\textcolor[rgb]{0.73,0.40,0.53}{{#1}}}
    \newcommand{\ImportTok}[1]{{#1}}
    \newcommand{\DocumentationTok}[1]{\textcolor[rgb]{0.73,0.13,0.13}{\textit{{#1}}}}
    \newcommand{\AnnotationTok}[1]{\textcolor[rgb]{0.38,0.63,0.69}{\textbf{\textit{{#1}}}}}
    \newcommand{\CommentVarTok}[1]{\textcolor[rgb]{0.38,0.63,0.69}{\textbf{\textit{{#1}}}}}
    \newcommand{\VariableTok}[1]{\textcolor[rgb]{0.10,0.09,0.49}{{#1}}}
    \newcommand{\ControlFlowTok}[1]{\textcolor[rgb]{0.00,0.44,0.13}{\textbf{{#1}}}}
    \newcommand{\OperatorTok}[1]{\textcolor[rgb]{0.40,0.40,0.40}{{#1}}}
    \newcommand{\BuiltInTok}[1]{{#1}}
    \newcommand{\ExtensionTok}[1]{{#1}}
    \newcommand{\PreprocessorTok}[1]{\textcolor[rgb]{0.74,0.48,0.00}{{#1}}}
    \newcommand{\AttributeTok}[1]{\textcolor[rgb]{0.49,0.56,0.16}{{#1}}}
    \newcommand{\InformationTok}[1]{\textcolor[rgb]{0.38,0.63,0.69}{\textbf{\textit{{#1}}}}}
    \newcommand{\WarningTok}[1]{\textcolor[rgb]{0.38,0.63,0.69}{\textbf{\textit{{#1}}}}}
    
    
    % Define a nice break command that doesn't care if a line doesn't already
    % exist.
    \def\br{\hspace*{\fill} \\* }
    % Math Jax compatability definitions
    \def\gt{>}
    \def\lt{<}
    % Document parameters
    \title{XGBoost\_Whales}
    
    
    

    % Pygments definitions
    
\makeatletter
\def\PY@reset{\let\PY@it=\relax \let\PY@bf=\relax%
    \let\PY@ul=\relax \let\PY@tc=\relax%
    \let\PY@bc=\relax \let\PY@ff=\relax}
\def\PY@tok#1{\csname PY@tok@#1\endcsname}
\def\PY@toks#1+{\ifx\relax#1\empty\else%
    \PY@tok{#1}\expandafter\PY@toks\fi}
\def\PY@do#1{\PY@bc{\PY@tc{\PY@ul{%
    \PY@it{\PY@bf{\PY@ff{#1}}}}}}}
\def\PY#1#2{\PY@reset\PY@toks#1+\relax+\PY@do{#2}}

\expandafter\def\csname PY@tok@w\endcsname{\def\PY@tc##1{\textcolor[rgb]{0.73,0.73,0.73}{##1}}}
\expandafter\def\csname PY@tok@c\endcsname{\let\PY@it=\textit\def\PY@tc##1{\textcolor[rgb]{0.25,0.50,0.50}{##1}}}
\expandafter\def\csname PY@tok@cp\endcsname{\def\PY@tc##1{\textcolor[rgb]{0.74,0.48,0.00}{##1}}}
\expandafter\def\csname PY@tok@k\endcsname{\let\PY@bf=\textbf\def\PY@tc##1{\textcolor[rgb]{0.00,0.50,0.00}{##1}}}
\expandafter\def\csname PY@tok@kp\endcsname{\def\PY@tc##1{\textcolor[rgb]{0.00,0.50,0.00}{##1}}}
\expandafter\def\csname PY@tok@kt\endcsname{\def\PY@tc##1{\textcolor[rgb]{0.69,0.00,0.25}{##1}}}
\expandafter\def\csname PY@tok@o\endcsname{\def\PY@tc##1{\textcolor[rgb]{0.40,0.40,0.40}{##1}}}
\expandafter\def\csname PY@tok@ow\endcsname{\let\PY@bf=\textbf\def\PY@tc##1{\textcolor[rgb]{0.67,0.13,1.00}{##1}}}
\expandafter\def\csname PY@tok@nb\endcsname{\def\PY@tc##1{\textcolor[rgb]{0.00,0.50,0.00}{##1}}}
\expandafter\def\csname PY@tok@nf\endcsname{\def\PY@tc##1{\textcolor[rgb]{0.00,0.00,1.00}{##1}}}
\expandafter\def\csname PY@tok@nc\endcsname{\let\PY@bf=\textbf\def\PY@tc##1{\textcolor[rgb]{0.00,0.00,1.00}{##1}}}
\expandafter\def\csname PY@tok@nn\endcsname{\let\PY@bf=\textbf\def\PY@tc##1{\textcolor[rgb]{0.00,0.00,1.00}{##1}}}
\expandafter\def\csname PY@tok@ne\endcsname{\let\PY@bf=\textbf\def\PY@tc##1{\textcolor[rgb]{0.82,0.25,0.23}{##1}}}
\expandafter\def\csname PY@tok@nv\endcsname{\def\PY@tc##1{\textcolor[rgb]{0.10,0.09,0.49}{##1}}}
\expandafter\def\csname PY@tok@no\endcsname{\def\PY@tc##1{\textcolor[rgb]{0.53,0.00,0.00}{##1}}}
\expandafter\def\csname PY@tok@nl\endcsname{\def\PY@tc##1{\textcolor[rgb]{0.63,0.63,0.00}{##1}}}
\expandafter\def\csname PY@tok@ni\endcsname{\let\PY@bf=\textbf\def\PY@tc##1{\textcolor[rgb]{0.60,0.60,0.60}{##1}}}
\expandafter\def\csname PY@tok@na\endcsname{\def\PY@tc##1{\textcolor[rgb]{0.49,0.56,0.16}{##1}}}
\expandafter\def\csname PY@tok@nt\endcsname{\let\PY@bf=\textbf\def\PY@tc##1{\textcolor[rgb]{0.00,0.50,0.00}{##1}}}
\expandafter\def\csname PY@tok@nd\endcsname{\def\PY@tc##1{\textcolor[rgb]{0.67,0.13,1.00}{##1}}}
\expandafter\def\csname PY@tok@s\endcsname{\def\PY@tc##1{\textcolor[rgb]{0.73,0.13,0.13}{##1}}}
\expandafter\def\csname PY@tok@sd\endcsname{\let\PY@it=\textit\def\PY@tc##1{\textcolor[rgb]{0.73,0.13,0.13}{##1}}}
\expandafter\def\csname PY@tok@si\endcsname{\let\PY@bf=\textbf\def\PY@tc##1{\textcolor[rgb]{0.73,0.40,0.53}{##1}}}
\expandafter\def\csname PY@tok@se\endcsname{\let\PY@bf=\textbf\def\PY@tc##1{\textcolor[rgb]{0.73,0.40,0.13}{##1}}}
\expandafter\def\csname PY@tok@sr\endcsname{\def\PY@tc##1{\textcolor[rgb]{0.73,0.40,0.53}{##1}}}
\expandafter\def\csname PY@tok@ss\endcsname{\def\PY@tc##1{\textcolor[rgb]{0.10,0.09,0.49}{##1}}}
\expandafter\def\csname PY@tok@sx\endcsname{\def\PY@tc##1{\textcolor[rgb]{0.00,0.50,0.00}{##1}}}
\expandafter\def\csname PY@tok@m\endcsname{\def\PY@tc##1{\textcolor[rgb]{0.40,0.40,0.40}{##1}}}
\expandafter\def\csname PY@tok@gh\endcsname{\let\PY@bf=\textbf\def\PY@tc##1{\textcolor[rgb]{0.00,0.00,0.50}{##1}}}
\expandafter\def\csname PY@tok@gu\endcsname{\let\PY@bf=\textbf\def\PY@tc##1{\textcolor[rgb]{0.50,0.00,0.50}{##1}}}
\expandafter\def\csname PY@tok@gd\endcsname{\def\PY@tc##1{\textcolor[rgb]{0.63,0.00,0.00}{##1}}}
\expandafter\def\csname PY@tok@gi\endcsname{\def\PY@tc##1{\textcolor[rgb]{0.00,0.63,0.00}{##1}}}
\expandafter\def\csname PY@tok@gr\endcsname{\def\PY@tc##1{\textcolor[rgb]{1.00,0.00,0.00}{##1}}}
\expandafter\def\csname PY@tok@ge\endcsname{\let\PY@it=\textit}
\expandafter\def\csname PY@tok@gs\endcsname{\let\PY@bf=\textbf}
\expandafter\def\csname PY@tok@gp\endcsname{\let\PY@bf=\textbf\def\PY@tc##1{\textcolor[rgb]{0.00,0.00,0.50}{##1}}}
\expandafter\def\csname PY@tok@go\endcsname{\def\PY@tc##1{\textcolor[rgb]{0.53,0.53,0.53}{##1}}}
\expandafter\def\csname PY@tok@gt\endcsname{\def\PY@tc##1{\textcolor[rgb]{0.00,0.27,0.87}{##1}}}
\expandafter\def\csname PY@tok@err\endcsname{\def\PY@bc##1{\setlength{\fboxsep}{0pt}\fcolorbox[rgb]{1.00,0.00,0.00}{1,1,1}{\strut ##1}}}
\expandafter\def\csname PY@tok@kc\endcsname{\let\PY@bf=\textbf\def\PY@tc##1{\textcolor[rgb]{0.00,0.50,0.00}{##1}}}
\expandafter\def\csname PY@tok@kd\endcsname{\let\PY@bf=\textbf\def\PY@tc##1{\textcolor[rgb]{0.00,0.50,0.00}{##1}}}
\expandafter\def\csname PY@tok@kn\endcsname{\let\PY@bf=\textbf\def\PY@tc##1{\textcolor[rgb]{0.00,0.50,0.00}{##1}}}
\expandafter\def\csname PY@tok@kr\endcsname{\let\PY@bf=\textbf\def\PY@tc##1{\textcolor[rgb]{0.00,0.50,0.00}{##1}}}
\expandafter\def\csname PY@tok@bp\endcsname{\def\PY@tc##1{\textcolor[rgb]{0.00,0.50,0.00}{##1}}}
\expandafter\def\csname PY@tok@fm\endcsname{\def\PY@tc##1{\textcolor[rgb]{0.00,0.00,1.00}{##1}}}
\expandafter\def\csname PY@tok@vc\endcsname{\def\PY@tc##1{\textcolor[rgb]{0.10,0.09,0.49}{##1}}}
\expandafter\def\csname PY@tok@vg\endcsname{\def\PY@tc##1{\textcolor[rgb]{0.10,0.09,0.49}{##1}}}
\expandafter\def\csname PY@tok@vi\endcsname{\def\PY@tc##1{\textcolor[rgb]{0.10,0.09,0.49}{##1}}}
\expandafter\def\csname PY@tok@vm\endcsname{\def\PY@tc##1{\textcolor[rgb]{0.10,0.09,0.49}{##1}}}
\expandafter\def\csname PY@tok@sa\endcsname{\def\PY@tc##1{\textcolor[rgb]{0.73,0.13,0.13}{##1}}}
\expandafter\def\csname PY@tok@sb\endcsname{\def\PY@tc##1{\textcolor[rgb]{0.73,0.13,0.13}{##1}}}
\expandafter\def\csname PY@tok@sc\endcsname{\def\PY@tc##1{\textcolor[rgb]{0.73,0.13,0.13}{##1}}}
\expandafter\def\csname PY@tok@dl\endcsname{\def\PY@tc##1{\textcolor[rgb]{0.73,0.13,0.13}{##1}}}
\expandafter\def\csname PY@tok@s2\endcsname{\def\PY@tc##1{\textcolor[rgb]{0.73,0.13,0.13}{##1}}}
\expandafter\def\csname PY@tok@sh\endcsname{\def\PY@tc##1{\textcolor[rgb]{0.73,0.13,0.13}{##1}}}
\expandafter\def\csname PY@tok@s1\endcsname{\def\PY@tc##1{\textcolor[rgb]{0.73,0.13,0.13}{##1}}}
\expandafter\def\csname PY@tok@mb\endcsname{\def\PY@tc##1{\textcolor[rgb]{0.40,0.40,0.40}{##1}}}
\expandafter\def\csname PY@tok@mf\endcsname{\def\PY@tc##1{\textcolor[rgb]{0.40,0.40,0.40}{##1}}}
\expandafter\def\csname PY@tok@mh\endcsname{\def\PY@tc##1{\textcolor[rgb]{0.40,0.40,0.40}{##1}}}
\expandafter\def\csname PY@tok@mi\endcsname{\def\PY@tc##1{\textcolor[rgb]{0.40,0.40,0.40}{##1}}}
\expandafter\def\csname PY@tok@il\endcsname{\def\PY@tc##1{\textcolor[rgb]{0.40,0.40,0.40}{##1}}}
\expandafter\def\csname PY@tok@mo\endcsname{\def\PY@tc##1{\textcolor[rgb]{0.40,0.40,0.40}{##1}}}
\expandafter\def\csname PY@tok@ch\endcsname{\let\PY@it=\textit\def\PY@tc##1{\textcolor[rgb]{0.25,0.50,0.50}{##1}}}
\expandafter\def\csname PY@tok@cm\endcsname{\let\PY@it=\textit\def\PY@tc##1{\textcolor[rgb]{0.25,0.50,0.50}{##1}}}
\expandafter\def\csname PY@tok@cpf\endcsname{\let\PY@it=\textit\def\PY@tc##1{\textcolor[rgb]{0.25,0.50,0.50}{##1}}}
\expandafter\def\csname PY@tok@c1\endcsname{\let\PY@it=\textit\def\PY@tc##1{\textcolor[rgb]{0.25,0.50,0.50}{##1}}}
\expandafter\def\csname PY@tok@cs\endcsname{\let\PY@it=\textit\def\PY@tc##1{\textcolor[rgb]{0.25,0.50,0.50}{##1}}}

\def\PYZbs{\char`\\}
\def\PYZus{\char`\_}
\def\PYZob{\char`\{}
\def\PYZcb{\char`\}}
\def\PYZca{\char`\^}
\def\PYZam{\char`\&}
\def\PYZlt{\char`\<}
\def\PYZgt{\char`\>}
\def\PYZsh{\char`\#}
\def\PYZpc{\char`\%}
\def\PYZdl{\char`\$}
\def\PYZhy{\char`\-}
\def\PYZsq{\char`\'}
\def\PYZdq{\char`\"}
\def\PYZti{\char`\~}
% for compatibility with earlier versions
\def\PYZat{@}
\def\PYZlb{[}
\def\PYZrb{]}
\makeatother


    % Exact colors from NB
    \definecolor{incolor}{rgb}{0.0, 0.0, 0.5}
    \definecolor{outcolor}{rgb}{0.545, 0.0, 0.0}



    
    % Prevent overflowing lines due to hard-to-break entities
    \sloppy 
    % Setup hyperref package
    \hypersetup{
      breaklinks=true,  % so long urls are correctly broken across lines
      colorlinks=true,
      urlcolor=urlcolor,
      linkcolor=linkcolor,
      citecolor=citecolor,
      }
    % Slightly bigger margins than the latex defaults
    
    \geometry{verbose,tmargin=1in,bmargin=1in,lmargin=1in,rmargin=1in}
    
    

    \begin{document}
    
    
    \maketitle
    
    

    
    \subsection{Problem Statment}\label{problem-statment}

You have been provided a dataset that consists of echo-location clicks
of two types of whales, namely, Gervais and Cuviers. In this assignment,
your task is to classify the different types whales using Gradient
Boosting with the help of the XGBoost library. You are expected to fill
in functions that would complete this task. We use XGBoost here instead
of GradientBoostedTrees in Spark because XGBoost running on a single
machine is much faster than Spark running on 10 machines.

The data files were preprocessed on PySpark (10 nodes) cluster. The code
for the same can be found in Data\_Processing\_Whales.ipynb. The
preprocessed data is a numpy array with \texttt{4175} rows (for the 10mb
file) with following columns (zero-indexed): * Col 0-9: projections on
first 10 eigen vectors * Col 10: rmse * Col 11: peak2peak * Col 12:
label
(\texttt{0\ if\ row.species==u\textquotesingle{}Gervais\textquotesingle{}\ else\ 1})

You can also refer to XGBoost\_Whales.ipynb under for more details on
the XGBoost Analysis before you attempt this assignment.

Both Data\_Processing\_Whales.ipynb and XGBoost\_Whales.ipynb can be
found under XGBoost directory that was uploaded in edX as a part of
"Notebooks for weeks 7 \& 8".

    \subsection{XGBoost - Theory}\label{xgboost---theory}

A brief overview of gradient boosting in XGBoost can be found here:

\begin{itemize}
\tightlist
\item
  http://xgboost.readthedocs.io/en/latest/model.html
\item
  http://xgboost.readthedocs.io/en/latest/python/python\_intro.html
\end{itemize}

    Use the XGBoost API for training and predicting scores:

\begin{itemize}
\tightlist
\item
  http://xgboost.readthedocs.io/en/latest/python/python\_api.html
\end{itemize}

\paragraph{Main API}\label{main-api}

\begin{itemize}
\tightlist
\item
  \texttt{xgboost.train} is the learning API that trains the Gradient
  Boosting Model,
\item
  The main parameters are:

  \begin{itemize}
  \tightlist
  \item
    \textbf{plst} -- XGBoost parameter list
  \item
    \textbf{dtrain} -- Data to be trained
  \item
    \textbf{num\_round} -- Number of iterations of boosting. (default:
    100)
  \item
    \textbf{evallist} -- List of items to be evaluated during training
  \item
    \textbf{verbose\_eval} - This can be used to control how much
    information the train function prints. You might want to set to
    \textbf{False} to avoid printing logs.
  \end{itemize}
\item
  \texttt{bst.predict} is the API that makes score predictions
\item
  The main parameters are:

  \begin{itemize}
  \tightlist
  \item
    \textbf{dtest} -- Test Data
  \item
    \textbf{dtrain} -- Data to be trained
  \item
    \textbf{ntree\_limit} -- Limit number of trees in the prediction
    (Use: ntree\_limit=bst.best\_ntree\_limit)
  \item
    \textbf{output\_margin} - Whether to output the raw untransformed
    margin value (Use: output\_margin=True)
  \end{itemize}
\end{itemize}

    \subsection{Notebook Setup}\label{notebook-setup}

    \subsubsection{Importing Required
Libraries}\label{importing-required-libraries}

    \begin{Verbatim}[commandchars=\\\{\}]
{\color{incolor}In [{\color{incolor}1}]:} \PY{o}{\PYZpc{}}\PY{k}{matplotlib} inline
        \PY{k+kn}{import} \PY{n+nn}{numpy} \PY{k}{as} \PY{n+nn}{np}
        \PY{k+kn}{import} \PY{n+nn}{xgboost} \PY{k}{as} \PY{n+nn}{xgb}
        \PY{k+kn}{from} \PY{n+nn}{sklearn}\PY{n+nn}{.}\PY{n+nn}{model\PYZus{}selection} \PY{k}{import} \PY{n}{train\PYZus{}test\PYZus{}split}
        \PY{k+kn}{import} \PY{n+nn}{matplotlib}\PY{n+nn}{.}\PY{n+nn}{pyplot} \PY{k}{as} \PY{n+nn}{plt}
        \PY{k+kn}{import} \PY{n+nn}{pickle}
        \PY{k+kn}{import} \PY{n+nn}{random}
\end{Verbatim}


    \subsubsection{Loading Data}\label{loading-data}

    \begin{Verbatim}[commandchars=\\\{\}]
{\color{incolor}In [{\color{incolor}2}]:} \PY{k}{with} \PY{n+nb}{open}\PY{p}{(}\PY{l+s+s1}{\PYZsq{}}\PY{l+s+s1}{Data/X\PYZus{}train.pkl}\PY{l+s+s1}{\PYZsq{}}\PY{p}{,} \PY{l+s+s1}{\PYZsq{}}\PY{l+s+s1}{rb}\PY{l+s+s1}{\PYZsq{}}\PY{p}{)} \PY{k}{as} \PY{n}{f}\PY{p}{:}
            \PY{n}{X\PYZus{}train} \PY{o}{=} \PY{n}{pickle}\PY{o}{.}\PY{n}{load}\PY{p}{(}\PY{n}{f}\PY{p}{)}
        
        \PY{k}{with} \PY{n+nb}{open}\PY{p}{(}\PY{l+s+s1}{\PYZsq{}}\PY{l+s+s1}{Data/X\PYZus{}test.pkl}\PY{l+s+s1}{\PYZsq{}}\PY{p}{,} \PY{l+s+s1}{\PYZsq{}}\PY{l+s+s1}{rb}\PY{l+s+s1}{\PYZsq{}}\PY{p}{)} \PY{k}{as} \PY{n}{f}\PY{p}{:}
            \PY{n}{X\PYZus{}test} \PY{o}{=} \PY{n}{pickle}\PY{o}{.}\PY{n}{load}\PY{p}{(}\PY{n}{f}\PY{p}{)}
        
        \PY{k}{with} \PY{n+nb}{open}\PY{p}{(}\PY{l+s+s1}{\PYZsq{}}\PY{l+s+s1}{Data/y\PYZus{}train.pkl}\PY{l+s+s1}{\PYZsq{}}\PY{p}{,} \PY{l+s+s1}{\PYZsq{}}\PY{l+s+s1}{rb}\PY{l+s+s1}{\PYZsq{}}\PY{p}{)} \PY{k}{as} \PY{n}{f}\PY{p}{:}
            \PY{n}{y\PYZus{}train} \PY{o}{=} \PY{n}{pickle}\PY{o}{.}\PY{n}{load}\PY{p}{(}\PY{n}{f}\PY{p}{)}
        
        \PY{k}{with} \PY{n+nb}{open}\PY{p}{(}\PY{l+s+s1}{\PYZsq{}}\PY{l+s+s1}{Data/y\PYZus{}test.pkl}\PY{l+s+s1}{\PYZsq{}}\PY{p}{,} \PY{l+s+s1}{\PYZsq{}}\PY{l+s+s1}{rb}\PY{l+s+s1}{\PYZsq{}}\PY{p}{)} \PY{k}{as} \PY{n}{f}\PY{p}{:}
            \PY{n}{y\PYZus{}test} \PY{o}{=} \PY{n}{pickle}\PY{o}{.}\PY{n}{load}\PY{p}{(}\PY{n}{f}\PY{p}{)}
\end{Verbatim}


    \subsubsection{Setting Parameters for XG
Boost}\label{setting-parameters-for-xg-boost}

\begin{itemize}
\tightlist
\item
  Maximum Depth of the Tree = 3 \emph{(maximum depth of each decision
  trees)}
\item
  Step size shrinkage used in update to prevents overfitting = 0.3
  \emph{(how to weigh trees in subsequent iterations)}
\item
  Evaluation Criterion= Maximize Loglikelihood according to the logistic
  regression \emph{(logitboost)}
\item
  Maximum Number of Iterations = 1000 \emph{(total number trees for
  boosting)}
\item
  Early Stop if score on Validation does not improve for 5 iterations
\end{itemize}

\href{https://xgboost.readthedocs.io/en/latest//parameter.html}{Full
description of options}

    \begin{Verbatim}[commandchars=\\\{\}]
{\color{incolor}In [{\color{incolor}3}]:} \PY{c+c1}{\PYZsh{}You can change this cell if you wish to, but you aren\PYZsq{}t expected to}
        \PY{k}{def} \PY{n+nf}{xgboost\PYZus{}plst}\PY{p}{(}\PY{p}{)}\PY{p}{:}
            \PY{n}{param} \PY{o}{=} \PY{p}{\PYZob{}}\PY{p}{\PYZcb{}}
            \PY{n}{param}\PY{p}{[}\PY{l+s+s1}{\PYZsq{}}\PY{l+s+s1}{max\PYZus{}depth}\PY{l+s+s1}{\PYZsq{}}\PY{p}{]}\PY{o}{=} \PY{l+m+mi}{3}   \PY{c+c1}{\PYZsh{} depth of tree}
            \PY{n}{param}\PY{p}{[}\PY{l+s+s1}{\PYZsq{}}\PY{l+s+s1}{eta}\PY{l+s+s1}{\PYZsq{}}\PY{p}{]} \PY{o}{=} \PY{l+m+mf}{0.3}      \PY{c+c1}{\PYZsh{} shrinkage parameter}
            \PY{n}{param}\PY{p}{[}\PY{l+s+s1}{\PYZsq{}}\PY{l+s+s1}{silent}\PY{l+s+s1}{\PYZsq{}}\PY{p}{]} \PY{o}{=} \PY{l+m+mi}{1}     \PY{c+c1}{\PYZsh{} not silent}
            \PY{n}{param}\PY{p}{[}\PY{l+s+s1}{\PYZsq{}}\PY{l+s+s1}{objective}\PY{l+s+s1}{\PYZsq{}}\PY{p}{]} \PY{o}{=} \PY{l+s+s1}{\PYZsq{}}\PY{l+s+s1}{binary:logistic}\PY{l+s+s1}{\PYZsq{}}
            \PY{n}{param}\PY{p}{[}\PY{l+s+s1}{\PYZsq{}}\PY{l+s+s1}{nthread}\PY{l+s+s1}{\PYZsq{}}\PY{p}{]} \PY{o}{=} \PY{l+m+mi}{7} \PY{c+c1}{\PYZsh{} Number of threads used}
            \PY{n}{param}\PY{p}{[}\PY{l+s+s1}{\PYZsq{}}\PY{l+s+s1}{eval\PYZus{}metric}\PY{l+s+s1}{\PYZsq{}}\PY{p}{]} \PY{o}{=} \PY{l+s+s1}{\PYZsq{}}\PY{l+s+s1}{logloss}\PY{l+s+s1}{\PYZsq{}}
        
            \PY{n}{plst} \PY{o}{=} \PY{n}{param}\PY{o}{.}\PY{n}{items}\PY{p}{(}\PY{p}{)}
            \PY{k}{return} \PY{n}{plst}
\end{Verbatim}


    \subsection{Exercises}\label{exercises}

    \subsubsection{Computing the score
ranges}\label{computing-the-score-ranges}

The function calc\_stats takes the xgboost margin scores as input and
returns two numpy arrays min\_scr, max\_scr which are calculated as
follows:

\begin{enumerate}
\def\labelenumi{\arabic{enumi}.}
\tightlist
\item
  \textbf{min\_scr}: mean - (3 \(\times\) std)
\item
  \textbf{max\_scr}: mean + (3 \(\times\) std)
\end{enumerate}

Here the input margin scores, represents the processed XGBoost margin
scores obtained from the bootstrap\_pred function. Each row represents
the various scores for a specific example in an iteration and your
calc\_stats function is supposed to compute the \textbf{min\_scr} and
\textbf{max\_scr} as defined for each example. So in the example below,
we take a scenario where we have 3 examples which have 4 values each
(From 4 bootstrap iterations).

\textbf{Example Input}

\begin{Shaded}
\begin{Highlighting}[]
\NormalTok{[[}\OperatorTok{-}\FloatTok{0.22} \OperatorTok{-}\FloatTok{0.19} \OperatorTok{-}\FloatTok{0.17} \OperatorTok{-}\FloatTok{0.13}\NormalTok{][}\OperatorTok{-}\FloatTok{0.1} \OperatorTok{-}\FloatTok{0.05} \FloatTok{0.02} \FloatTok{0.10}\NormalTok{][}\FloatTok{0.03} \FloatTok{0.11} \FloatTok{0.12} \FloatTok{0.15}\NormalTok{]]}
\end{Highlighting}
\end{Shaded}

Output: min\_scr (numpy array), max\_scr (numpy array)

\textbf{Example Output}

\begin{Shaded}
\begin{Highlighting}[]
\NormalTok{(array([}\OperatorTok{-}\FloatTok{0.28} \OperatorTok{-}\FloatTok{0.23} \OperatorTok{-}\FloatTok{0.03}\NormalTok{]),}
\NormalTok{ array([}\OperatorTok{-}\FloatTok{0.08}  \FloatTok{0.22}  \FloatTok{0.24}\NormalTok{]))}
\end{Highlighting}
\end{Shaded}

\textbf{Note}: Ensure you round the values in the numpy arrays to two
decimal places

    \begin{Verbatim}[commandchars=\\\{\}]
{\color{incolor}In [{\color{incolor}4}]:} \PY{k}{def} \PY{n+nf}{calc\PYZus{}stats}\PY{p}{(}\PY{n}{margin\PYZus{}scores}\PY{p}{)}\PY{p}{:}
            \PY{c+c1}{\PYZsh{}}
            \PY{c+c1}{\PYZsh{} YOUR CODE HERE}
            \PY{c+c1}{\PYZsh{}}
\end{Verbatim}


    \begin{Verbatim}[commandchars=\\\{\}]
{\color{incolor}In [{\color{incolor}5}]:} \PY{n}{margin\PYZus{}score} \PY{o}{=} \PY{n}{np}\PY{o}{.}\PY{n}{array}\PY{p}{(}\PY{p}{[}\PY{p}{[}\PY{o}{\PYZhy{}}\PY{l+m+mf}{0.22}\PY{p}{,} \PY{o}{\PYZhy{}}\PY{l+m+mf}{0.19}\PY{p}{,} \PY{o}{\PYZhy{}}\PY{l+m+mf}{0.17}\PY{p}{,} \PY{o}{\PYZhy{}}\PY{l+m+mf}{0.13}\PY{p}{]}\PY{p}{,} \PY{p}{[}\PY{o}{\PYZhy{}}\PY{l+m+mf}{0.1}\PY{p}{,} \PY{o}{\PYZhy{}}\PY{l+m+mf}{0.05}\PY{p}{,} \PY{l+m+mf}{0.02}\PY{p}{,} \PY{l+m+mf}{0.10}\PY{p}{]}\PY{p}{,} \PY{p}{[}\PY{l+m+mf}{0.03}\PY{p}{,} \PY{l+m+mf}{0.11}\PY{p}{,} \PY{l+m+mf}{0.12}\PY{p}{,} \PY{l+m+mf}{0.15}\PY{p}{]}\PY{p}{]}\PY{p}{)}
        \PY{n}{min\PYZus{}score}\PY{p}{,} \PY{n}{max\PYZus{}score} \PY{o}{=} \PY{n}{calc\PYZus{}stats}\PY{p}{(}\PY{n}{margin\PYZus{}score}\PY{p}{)}
        \PY{k}{assert} \PY{n+nb}{type}\PY{p}{(}\PY{n}{min\PYZus{}score}\PY{p}{)} \PY{o}{==} \PY{n}{np}\PY{o}{.}\PY{n}{ndarray}\PY{p}{,} \PY{l+s+s1}{\PYZsq{}}\PY{l+s+s1}{Incorrect Return type}\PY{l+s+s1}{\PYZsq{}}
        \PY{k}{assert} \PY{n+nb}{type}\PY{p}{(}\PY{n}{max\PYZus{}score}\PY{p}{)} \PY{o}{==} \PY{n}{np}\PY{o}{.}\PY{n}{ndarray}\PY{p}{,} \PY{l+s+s1}{\PYZsq{}}\PY{l+s+s1}{Incorrect Return type}\PY{l+s+s1}{\PYZsq{}}
\end{Verbatim}


    \begin{Verbatim}[commandchars=\\\{\}]
{\color{incolor}In [{\color{incolor}6}]:} \PY{k}{assert} \PY{p}{(}\PY{n}{min\PYZus{}score} \PY{o}{==} \PY{n}{np}\PY{o}{.}\PY{n}{array}\PY{p}{(}\PY{p}{[}\PY{o}{\PYZhy{}}\PY{l+m+mf}{0.28}\PY{p}{,} \PY{o}{\PYZhy{}}\PY{l+m+mf}{0.23}\PY{p}{,} \PY{o}{\PYZhy{}}\PY{l+m+mf}{0.03}\PY{p}{]}\PY{p}{)}\PY{p}{)}\PY{o}{.}\PY{n}{all}\PY{p}{(}\PY{p}{)}\PY{p}{,} \PY{l+s+s2}{\PYZdq{}}\PY{l+s+s2}{Incorrect return value}\PY{l+s+s2}{\PYZdq{}}
\end{Verbatim}


    \begin{Verbatim}[commandchars=\\\{\}]
{\color{incolor}In [{\color{incolor}7}]:} \PY{k}{assert} \PY{p}{(}\PY{n}{max\PYZus{}score} \PY{o}{==} \PY{n}{np}\PY{o}{.}\PY{n}{array}\PY{p}{(}\PY{p}{[}\PY{o}{\PYZhy{}}\PY{l+m+mf}{0.08}\PY{p}{,}  \PY{l+m+mf}{0.22}\PY{p}{,}  \PY{l+m+mf}{0.24}\PY{p}{]}\PY{p}{)}\PY{p}{)}\PY{o}{.}\PY{n}{all}\PY{p}{(}\PY{p}{)}\PY{p}{,} \PY{l+s+s2}{\PYZdq{}}\PY{l+s+s2}{Incorrect return value}\PY{l+s+s2}{\PYZdq{}}
\end{Verbatim}


    \begin{Verbatim}[commandchars=\\\{\}]
{\color{incolor}In [{\color{incolor}8}]:} \PY{c+c1}{\PYZsh{} Hidden Tests Here}
        \PY{c+c1}{\PYZsh{}}
        \PY{c+c1}{\PYZsh{} AUTOGRADER TEST \PYZhy{} DO NOT REMOVE}
        \PY{c+c1}{\PYZsh{}}
\end{Verbatim}


    \begin{Verbatim}[commandchars=\\\{\}]
{\color{incolor}In [{\color{incolor}9}]:} \PY{c+c1}{\PYZsh{} Hidden Tests Here}
        \PY{c+c1}{\PYZsh{}}
        \PY{c+c1}{\PYZsh{} AUTOGRADER TEST \PYZhy{} DO NOT REMOVE}
        \PY{c+c1}{\PYZsh{}}
\end{Verbatim}


    \subsubsection{Calculating predictions}\label{calculating-predictions}

    Based on the ranges for each of the examples, i.e, (min\_scr, max\_scr),
we can predict whether it's a Gervais or a Cuvier. Since all our scores
will be between -1 and +1, we use 0 as the margin line. All examples
which are on the left side of the margin, can be classified as Cuvier's
and all which are on the right side can be classified as Gervais.
However, since we take margin scores from a set of bootstraps for each
example, we use the minimum and maximum score arrays to predict to
determine the classification.

The function predict takes the minimum score array and maximum score
array and returns predictions as follows:

\begin{enumerate}
\def\labelenumi{\arabic{enumi}.}
\tightlist
\item
  If respective minimum score and maximum score values are less than 0,
  predict -1 (\textbf{Cuvier's})
\item
  If respective minimum score value is less than 0 and maximum score
  value is greater than 0, predict 0 (\textbf{Unsure})
\item
  If respective minimum score and maximum score values are greater than
  0, predict 1 (\textbf{Gervais})
\end{enumerate}

\textbf{Example Input}

\begin{Shaded}
\begin{Highlighting}[]
\NormalTok{min_scr (numpy array) }\OperatorTok{=}\NormalTok{ [}\OperatorTok{-}\FloatTok{0.78} \OperatorTok{-}\FloatTok{0.68} \OperatorTok{-}\FloatTok{0.6} \OperatorTok{-}\FloatTok{0.53} \OperatorTok{-}\FloatTok{0.47} \OperatorTok{-}\FloatTok{0.42} \OperatorTok{-}\FloatTok{0.32} \OperatorTok{-}\FloatTok{0.21} \OperatorTok{-}\FloatTok{0.07} \FloatTok{0.22}\NormalTok{]}

\NormalTok{max_scr (numpy array) }\OperatorTok{=}\NormalTok{ [}\OperatorTok{-}\FloatTok{0.49} \OperatorTok{-}\FloatTok{0.39} \OperatorTok{-}\FloatTok{0.33} \OperatorTok{-}\FloatTok{0.25} \OperatorTok{-}\FloatTok{0.2} \OperatorTok{-}\FloatTok{0.11} \OperatorTok{-}\FloatTok{0.04} \FloatTok{0.1} \FloatTok{0.3} \FloatTok{0.51}\NormalTok{]}
\end{Highlighting}
\end{Shaded}

Output: pred (numpy array of predictions)

\textbf{Example Output}

\begin{Shaded}
\begin{Highlighting}[]
\NormalTok{[}\OperatorTok{-}\DecValTok{1} \OperatorTok{-}\DecValTok{1} \OperatorTok{-}\DecValTok{1} \OperatorTok{-}\DecValTok{1} \OperatorTok{-}\DecValTok{1} \OperatorTok{-}\DecValTok{1} \OperatorTok{-}\DecValTok{1}  \DecValTok{0}  \DecValTok{0}  \DecValTok{1}\NormalTok{]}
\end{Highlighting}
\end{Shaded}

    \begin{Verbatim}[commandchars=\\\{\}]
{\color{incolor}In [{\color{incolor}10}]:} \PY{k}{def} \PY{n+nf}{predict}\PY{p}{(}\PY{n}{min\PYZus{}scr}\PY{p}{,} \PY{n}{max\PYZus{}scr}\PY{p}{)}\PY{p}{:}
             \PY{c+c1}{\PYZsh{}}
             \PY{c+c1}{\PYZsh{} YOUR CODE HERE}
             \PY{c+c1}{\PYZsh{}}
\end{Verbatim}


    \begin{Verbatim}[commandchars=\\\{\}]
{\color{incolor}In [{\color{incolor}11}]:} \PY{n}{max\PYZus{}s} \PY{o}{=} \PY{n}{np}\PY{o}{.}\PY{n}{array}\PY{p}{(}\PY{p}{[}\PY{o}{\PYZhy{}}\PY{l+m+mf}{0.49}\PY{p}{,} \PY{o}{\PYZhy{}}\PY{l+m+mf}{0.39}\PY{p}{,} \PY{o}{\PYZhy{}}\PY{l+m+mf}{0.33}\PY{p}{,} \PY{o}{\PYZhy{}}\PY{l+m+mf}{0.25}\PY{p}{,} \PY{o}{\PYZhy{}}\PY{l+m+mf}{0.2}\PY{p}{,} \PY{o}{\PYZhy{}}\PY{l+m+mf}{0.11}\PY{p}{,} \PY{o}{\PYZhy{}}\PY{l+m+mf}{0.04}\PY{p}{,} \PY{l+m+mf}{0.1}\PY{p}{,} \PY{l+m+mf}{0.3}\PY{p}{,} \PY{l+m+mf}{0.51}\PY{p}{]}\PY{p}{)}
         \PY{n}{min\PYZus{}s} \PY{o}{=} \PY{n}{np}\PY{o}{.}\PY{n}{array}\PY{p}{(}\PY{p}{[}\PY{o}{\PYZhy{}}\PY{l+m+mf}{0.78}\PY{p}{,} \PY{o}{\PYZhy{}}\PY{l+m+mf}{0.68}\PY{p}{,} \PY{o}{\PYZhy{}}\PY{l+m+mf}{0.6}\PY{p}{,} \PY{o}{\PYZhy{}}\PY{l+m+mf}{0.53}\PY{p}{,} \PY{o}{\PYZhy{}}\PY{l+m+mf}{0.47}\PY{p}{,} \PY{o}{\PYZhy{}}\PY{l+m+mf}{0.42}\PY{p}{,} \PY{o}{\PYZhy{}}\PY{l+m+mf}{0.32}\PY{p}{,} \PY{o}{\PYZhy{}}\PY{l+m+mf}{0.21}\PY{p}{,} \PY{o}{\PYZhy{}}\PY{l+m+mf}{0.07}\PY{p}{,} \PY{l+m+mf}{0.22}\PY{p}{]}\PY{p}{)}
         \PY{n}{pred} \PY{o}{=} \PY{n}{predict}\PY{p}{(}\PY{n}{min\PYZus{}s}\PY{p}{,} \PY{n}{max\PYZus{}s}\PY{p}{)}
         \PY{n}{true\PYZus{}pred} \PY{o}{=} \PY{n}{np}\PY{o}{.}\PY{n}{array}\PY{p}{(}\PY{p}{[}\PY{o}{\PYZhy{}}\PY{l+m+mi}{1}\PY{p}{,} \PY{o}{\PYZhy{}}\PY{l+m+mi}{1}\PY{p}{,} \PY{o}{\PYZhy{}}\PY{l+m+mi}{1}\PY{p}{,} \PY{o}{\PYZhy{}}\PY{l+m+mi}{1}\PY{p}{,} \PY{o}{\PYZhy{}}\PY{l+m+mi}{1}\PY{p}{,} \PY{o}{\PYZhy{}}\PY{l+m+mi}{1}\PY{p}{,} \PY{o}{\PYZhy{}}\PY{l+m+mi}{1}\PY{p}{,} \PY{l+m+mi}{0}\PY{p}{,} \PY{l+m+mi}{0}\PY{p}{,} \PY{l+m+mi}{1}\PY{p}{]}\PY{p}{)}
\end{Verbatim}


    \begin{Verbatim}[commandchars=\\\{\}]
{\color{incolor}In [{\color{incolor}12}]:} \PY{k}{assert} \PY{n+nb}{type}\PY{p}{(}\PY{n}{pred}\PY{p}{)} \PY{o}{==} \PY{n}{np}\PY{o}{.}\PY{n}{ndarray}\PY{p}{,} \PY{l+s+s1}{\PYZsq{}}\PY{l+s+s1}{Incorrect return type}\PY{l+s+s1}{\PYZsq{}}
\end{Verbatim}


    \begin{Verbatim}[commandchars=\\\{\}]
{\color{incolor}In [{\color{incolor}13}]:} \PY{k}{assert} \PY{p}{(}\PY{n}{pred} \PY{o}{==} \PY{n}{true\PYZus{}pred}\PY{p}{)}\PY{o}{.}\PY{n}{all}\PY{p}{(}\PY{p}{)}\PY{p}{,} \PY{l+s+s1}{\PYZsq{}}\PY{l+s+s1}{Incorrect return value}\PY{l+s+s1}{\PYZsq{}}
\end{Verbatim}


    \subsubsection{Calculating scores}\label{calculating-scores}

    The function bootstrap\_pred takes as input:

\begin{enumerate}
\def\labelenumi{\arabic{enumi}.}
\tightlist
\item
  \textbf{Training set}
\item
  \textbf{Test set}
\item
  \textbf{n\_bootstrap} Number of bootstrap samples that run XGBoost and
  trains one part of the sample set.
\item
  \textbf{minR, maxR} two numbers such that \(0 < minR < maxR < 1\) that
  define the fractions of the \texttt{n\_bootstrap} scores that define
  the range.
\item
  \textbf{bootstrap\_size} - Number of bootstrap samples on which you
  will run XGBoost.
\item
  \textbf{num\_round} - Number of iterations for running xgboost
\item
  \textbf{plst} - Parameter List
\end{enumerate}

The output should be a confidence interval for each example in the test
set. Together with a prediction that is
\texttt{Gervais\ /\ Cuviers\ /\ Unsure}. The prediction \texttt{unsure}
is to be output if the confidence interval contains the point 0. After
generating the confidence intervals, the function predict can be used to
make predictions.

    \textbf{Procedure}

Repeat the given procedure for n\_bootstrap number of iterations:

For \textbf{n\_bootstrap} iterations: * Sample \textbf{boostrap\_size}
indices from the training set \textbf{with replacmennt} * Create train
and test data matrices (dtrain, dtest) using xgb.DMatrix(X\_sample,
label=y\_sample) * Initialise the evallist parameter {[}(dtrain,
'train'), (dtest, 'eval'){]} * Train the model using the XGBoost train
API and make score predictions using bst.predict object returned by XGB
train API. Ensure you set \textbf{output\_margin=True} to get raw
untransformed output scores. * Normalize them by dividing them with the
normalizing factor as max(scores) - min(scores) and round these values
to a precision of two decimal places

Then: * For each individual example, remove scores below the minRth
percentile and greater than maxRth percentile (sort for each example if
necessary) * Call the calc\_stats function to compute min\_scr and
max\_scr with the filtered margin scores as parameter * Return the
min\_scr and max\_scr computed by the \textbf{calc\_stat} function using
the margin scores.

\textbf{Note}: You can experiment by changing \textbf{n\_bootstraps},
but it takes about 200 iterations to get consistent values.

    \begin{Verbatim}[commandchars=\\\{\}]
{\color{incolor}In [{\color{incolor}14}]:} \PY{k}{def} \PY{n+nf}{bootstrap\PYZus{}pred}\PY{p}{(}\PY{n}{X\PYZus{}train}\PY{p}{,} \PY{n}{X\PYZus{}test}\PY{p}{,} \PY{n}{y\PYZus{}train}\PY{p}{,} \PY{n}{y\PYZus{}test}\PY{p}{,} \PY{n}{n\PYZus{}bootstrap}\PY{p}{,} \PY{n}{minR}\PY{p}{,} \PY{n}{maxR}\PY{p}{,} \PY{n}{bootstrap\PYZus{}size}\PY{p}{,} \PYZbs{}
                            \PY{n}{num\PYZus{}round}\PY{o}{=}\PY{l+m+mi}{100}\PY{p}{,} \PY{n}{plst}\PY{o}{=}\PY{n}{xgboost\PYZus{}plst}\PY{p}{(}\PY{p}{)}\PY{p}{)}\PY{p}{:}
             \PY{c+c1}{\PYZsh{}}
             \PY{c+c1}{\PYZsh{} YOUR CODE HERE}
             \PY{c+c1}{\PYZsh{}}
\end{Verbatim}


    \begin{Verbatim}[commandchars=\\\{\}]
{\color{incolor}In [{\color{incolor}15}]:} \PY{k}{def} \PY{n+nf}{process}\PY{p}{(}\PY{n}{X\PYZus{}train}\PY{p}{,} \PY{n}{X\PYZus{}test}\PY{p}{,} \PY{n}{y\PYZus{}train}\PY{p}{,} \PY{n}{y\PYZus{}test}\PY{p}{,} \PY{n}{n\PYZus{}bootstrap}\PY{o}{=}\PY{l+m+mi}{100}\PY{p}{)}\PY{p}{:}
             \PY{n}{min\PYZus{}scr}\PY{p}{,} \PY{n}{max\PYZus{}scr} \PY{o}{=} \PY{n}{bootstrap\PYZus{}pred}\PY{p}{(}\PY{n}{X\PYZus{}train}\PY{p}{,} \PY{n}{X\PYZus{}test}\PY{p}{,} \PY{n}{y\PYZus{}train}\PY{p}{,} \PY{n}{y\PYZus{}test}\PY{p}{,} \PY{n}{n\PYZus{}bootstrap}\PY{o}{=}\PY{n}{n\PYZus{}bootstrap}\PY{p}{,} \PYZbs{}
                                                     \PY{n}{minR}\PY{o}{=}\PY{l+m+mf}{0.1}\PY{p}{,} \PY{n}{maxR}\PY{o}{=}\PY{l+m+mf}{0.9}\PY{p}{,} \PY{n}{bootstrap\PYZus{}size}\PY{o}{=}\PY{n+nb}{len}\PY{p}{(}\PY{n}{X\PYZus{}train}\PY{p}{)}\PY{p}{)}
             \PY{n}{pred} \PY{o}{=} \PY{n}{predict}\PY{p}{(}\PY{n}{min\PYZus{}scr}\PY{p}{,} \PY{n}{max\PYZus{}scr}\PY{p}{)}
             \PY{k}{return} \PY{n}{min\PYZus{}scr}\PY{p}{,} \PY{n}{max\PYZus{}scr}\PY{p}{,} \PY{n}{pred}
\end{Verbatim}


    \paragraph{Tests}\label{tests}

How we test the function: 1. We have computed the average mid-point of
the range of values and verify that this midpoint is present in the
range computed by your function 2. We check that the length of the
interval(max\_scr-min\_scr) is not more than twice the average length of
the interval

    \begin{Verbatim}[commandchars=\\\{\}]
{\color{incolor}In [{\color{incolor}16}]:} \PY{n}{sample\PYZus{}indices} \PY{o}{=} \PY{n}{np}\PY{o}{.}\PY{n}{load}\PY{p}{(}\PY{l+s+s1}{\PYZsq{}}\PY{l+s+s1}{Data/vis\PYZus{}indices.npy}\PY{l+s+s1}{\PYZsq{}}\PY{p}{)}
         \PY{n}{X\PYZus{}test\PYZus{}samp} \PY{o}{=} \PY{n}{X\PYZus{}test}\PY{p}{[}\PY{n}{sample\PYZus{}indices}\PY{p}{]}
         \PY{n}{y\PYZus{}test\PYZus{}samp} \PY{o}{=} \PY{n}{np}\PY{o}{.}\PY{n}{array}\PY{p}{(}\PY{n}{y\PYZus{}test}\PY{p}{[}\PY{n}{sample\PYZus{}indices}\PY{p}{]}\PY{p}{,} \PY{n}{dtype}\PY{o}{=}\PY{n+nb}{int}\PY{p}{)}
         \PY{n}{midpt} \PY{o}{=} \PY{n}{np}\PY{o}{.}\PY{n}{load}\PY{p}{(}\PY{l+s+s1}{\PYZsq{}}\PY{l+s+s1}{Data/vis\PYZus{}midpt.npy}\PY{l+s+s1}{\PYZsq{}}\PY{p}{)}
         \PY{n}{avg\PYZus{}length} \PY{o}{=} \PY{n}{np}\PY{o}{.}\PY{n}{load}\PY{p}{(}\PY{l+s+s1}{\PYZsq{}}\PY{l+s+s1}{Data/vis\PYZus{}avg\PYZus{}length.npy}\PY{l+s+s1}{\PYZsq{}}\PY{p}{)}
         \PY{n}{min\PYZus{}scr}\PY{p}{,} \PY{n}{max\PYZus{}scr}\PY{p}{,} \PY{n}{pred} \PY{o}{=} \PY{n}{process}\PY{p}{(}\PY{n}{X\PYZus{}train}\PY{p}{,} \PY{n}{X\PYZus{}test\PYZus{}samp}\PY{p}{,} \PY{n}{y\PYZus{}train}\PY{p}{,} \PY{n}{y\PYZus{}test\PYZus{}samp}\PY{p}{)}
         \PY{n}{length} \PY{o}{=} \PY{n}{max\PYZus{}scr} \PY{o}{\PYZhy{}} \PY{n}{min\PYZus{}scr}
\end{Verbatim}


    \begin{Verbatim}[commandchars=\\\{\}]
{\color{incolor}In [{\color{incolor}17}]:} \PY{k}{assert} \PY{n+nb}{sum}\PY{p}{(}\PY{n}{min\PYZus{}scr} \PY{o}{\PYZlt{}}\PY{o}{=} \PY{n}{midpt}\PY{p}{)} \PY{o}{\PYZgt{}}\PY{o}{=} \PY{p}{(}\PY{l+m+mf}{0.7} \PY{o}{*} \PY{n+nb}{len}\PY{p}{(}\PY{n}{sample\PYZus{}indices}\PY{p}{)}\PY{p}{)}\PY{p}{,} \PY{l+s+s2}{\PYZdq{}}\PY{l+s+s2}{Incorrect range (mean \PYZhy{} 3*std) to (mean + 3*std)}\PY{l+s+s2}{\PYZdq{}}
\end{Verbatim}


    \begin{Verbatim}[commandchars=\\\{\}]
{\color{incolor}In [{\color{incolor}18}]:} \PY{k}{assert} \PY{n+nb}{sum}\PY{p}{(}\PY{n}{max\PYZus{}scr} \PY{o}{\PYZgt{}}\PY{o}{=} \PY{n}{midpt}\PY{p}{)} \PY{o}{\PYZgt{}}\PY{o}{=} \PY{p}{(}\PY{l+m+mf}{0.7} \PY{o}{*} \PY{n+nb}{len}\PY{p}{(}\PY{n}{sample\PYZus{}indices}\PY{p}{)}\PY{p}{)}\PY{p}{,} \PY{l+s+s2}{\PYZdq{}}\PY{l+s+s2}{Incorrect range (mean \PYZhy{} 3*std) to (mean + 3*std)}\PY{l+s+s2}{\PYZdq{}}
\end{Verbatim}


    \begin{Verbatim}[commandchars=\\\{\}]
{\color{incolor}In [{\color{incolor}19}]:} \PY{k}{assert} \PY{n+nb}{sum}\PY{p}{(}\PY{n}{length} \PY{o}{\PYZlt{}} \PY{l+m+mi}{2}\PY{o}{*}\PY{n}{avg\PYZus{}length}\PY{p}{)} \PY{o}{\PYZgt{}}\PY{o}{=} \PY{p}{(}\PY{l+m+mf}{0.7} \PY{o}{*} \PY{n+nb}{len}\PY{p}{(}\PY{n}{sample\PYZus{}indices}\PY{p}{)}\PY{p}{)}\PY{p}{,} \PY{l+s+s2}{\PYZdq{}}\PY{l+s+s2}{Incorrect length of range (mean \PYZhy{} 3*std) to (mean + 3*std)}\PY{l+s+s2}{\PYZdq{}}
\end{Verbatim}


    \begin{Verbatim}[commandchars=\\\{\}]
{\color{incolor}In [{\color{incolor}22}]:} \PY{c+c1}{\PYZsh{} Hidden Tests Here}
         \PY{c+c1}{\PYZsh{}}
         \PY{c+c1}{\PYZsh{} AUTOGRADER TEST \PYZhy{} DO NOT REMOVE}
         \PY{c+c1}{\PYZsh{}}
\end{Verbatim}


    \begin{Verbatim}[commandchars=\\\{\}]
{\color{incolor}In [{\color{incolor}23}]:} \PY{c+c1}{\PYZsh{} Hidden Tests Here}
         \PY{c+c1}{\PYZsh{}}
         \PY{c+c1}{\PYZsh{} AUTOGRADER TEST \PYZhy{} DO NOT REMOVE}
         \PY{c+c1}{\PYZsh{}}
\end{Verbatim}


    \begin{Verbatim}[commandchars=\\\{\}]
{\color{incolor}In [{\color{incolor}24}]:} \PY{c+c1}{\PYZsh{} Hidden Tests Here}
         \PY{c+c1}{\PYZsh{}}
         \PY{c+c1}{\PYZsh{} AUTOGRADER TEST \PYZhy{} DO NOT REMOVE}
         \PY{c+c1}{\PYZsh{}}
\end{Verbatim}


    \begin{Verbatim}[commandchars=\\\{\}]
{\color{incolor}In [{\color{incolor}25}]:} \PY{c+c1}{\PYZsh{} Hidden Tests Here}
         \PY{c+c1}{\PYZsh{}}
         \PY{c+c1}{\PYZsh{} AUTOGRADER TEST \PYZhy{} DO NOT REMOVE}
         \PY{c+c1}{\PYZsh{}}
\end{Verbatim}


    \begin{Verbatim}[commandchars=\\\{\}]
{\color{incolor}In [{\color{incolor}27}]:} \PY{c+c1}{\PYZsh{} Hidden Tests here}
         \PY{c+c1}{\PYZsh{}}
         \PY{c+c1}{\PYZsh{} AUTOGRADER TEST \PYZhy{} DO NOT REMOVE}
         \PY{c+c1}{\PYZsh{}}
\end{Verbatim}


    \begin{Verbatim}[commandchars=\\\{\}]
{\color{incolor}In [{\color{incolor}28}]:} \PY{c+c1}{\PYZsh{} Hidden Tests here}
         \PY{c+c1}{\PYZsh{}}
         \PY{c+c1}{\PYZsh{} AUTOGRADER TEST \PYZhy{} DO NOT REMOVE}
         \PY{c+c1}{\PYZsh{}}
\end{Verbatim}


    \begin{Verbatim}[commandchars=\\\{\}]
{\color{incolor}In [{\color{incolor}29}]:} \PY{c+c1}{\PYZsh{} Hidden Tests here}
         \PY{c+c1}{\PYZsh{}}
         \PY{c+c1}{\PYZsh{} AUTOGRADER TEST \PYZhy{} DO NOT REMOVE}
         \PY{c+c1}{\PYZsh{}}
\end{Verbatim}


    \begin{Verbatim}[commandchars=\\\{\}]
{\color{incolor}In [{\color{incolor}30}]:} \PY{c+c1}{\PYZsh{} Hidden Tests here}
         \PY{c+c1}{\PYZsh{}}
         \PY{c+c1}{\PYZsh{} AUTOGRADER TEST \PYZhy{} DO NOT REMOVE}
         \PY{c+c1}{\PYZsh{}}
\end{Verbatim}



    % Add a bibliography block to the postdoc
    
    
    
    \end{document}
