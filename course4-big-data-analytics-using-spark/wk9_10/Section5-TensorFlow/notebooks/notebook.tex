
% Default to the notebook output style

    


% Inherit from the specified cell style.




    
\documentclass[11pt]{article}

    
    
    \usepackage[T1]{fontenc}
    % Nicer default font (+ math font) than Computer Modern for most use cases
    \usepackage{mathpazo}

    % Basic figure setup, for now with no caption control since it's done
    % automatically by Pandoc (which extracts ![](path) syntax from Markdown).
    \usepackage{graphicx}
    % We will generate all images so they have a width \maxwidth. This means
    % that they will get their normal width if they fit onto the page, but
    % are scaled down if they would overflow the margins.
    \makeatletter
    \def\maxwidth{\ifdim\Gin@nat@width>\linewidth\linewidth
    \else\Gin@nat@width\fi}
    \makeatother
    \let\Oldincludegraphics\includegraphics
    % Set max figure width to be 80% of text width, for now hardcoded.
    \renewcommand{\includegraphics}[1]{\Oldincludegraphics[width=.8\maxwidth]{#1}}
    % Ensure that by default, figures have no caption (until we provide a
    % proper Figure object with a Caption API and a way to capture that
    % in the conversion process - todo).
    \usepackage{caption}
    \DeclareCaptionLabelFormat{nolabel}{}
    \captionsetup{labelformat=nolabel}

    \usepackage{adjustbox} % Used to constrain images to a maximum size 
    \usepackage{xcolor} % Allow colors to be defined
    \usepackage{enumerate} % Needed for markdown enumerations to work
    \usepackage{geometry} % Used to adjust the document margins
    \usepackage{amsmath} % Equations
    \usepackage{amssymb} % Equations
    \usepackage{textcomp} % defines textquotesingle
    % Hack from http://tex.stackexchange.com/a/47451/13684:
    \AtBeginDocument{%
        \def\PYZsq{\textquotesingle}% Upright quotes in Pygmentized code
    }
    \usepackage{upquote} % Upright quotes for verbatim code
    \usepackage{eurosym} % defines \euro
    \usepackage[mathletters]{ucs} % Extended unicode (utf-8) support
    \usepackage[utf8x]{inputenc} % Allow utf-8 characters in the tex document
    \usepackage{fancyvrb} % verbatim replacement that allows latex
    \usepackage{grffile} % extends the file name processing of package graphics 
                         % to support a larger range 
    % The hyperref package gives us a pdf with properly built
    % internal navigation ('pdf bookmarks' for the table of contents,
    % internal cross-reference links, web links for URLs, etc.)
    \usepackage{hyperref}
    \usepackage{longtable} % longtable support required by pandoc >1.10
    \usepackage{booktabs}  % table support for pandoc > 1.12.2
    \usepackage[inline]{enumitem} % IRkernel/repr support (it uses the enumerate* environment)
    \usepackage[normalem]{ulem} % ulem is needed to support strikethroughs (\sout)
                                % normalem makes italics be italics, not underlines
    

    
    
    % Colors for the hyperref package
    \definecolor{urlcolor}{rgb}{0,.145,.698}
    \definecolor{linkcolor}{rgb}{.71,0.21,0.01}
    \definecolor{citecolor}{rgb}{.12,.54,.11}

    % ANSI colors
    \definecolor{ansi-black}{HTML}{3E424D}
    \definecolor{ansi-black-intense}{HTML}{282C36}
    \definecolor{ansi-red}{HTML}{E75C58}
    \definecolor{ansi-red-intense}{HTML}{B22B31}
    \definecolor{ansi-green}{HTML}{00A250}
    \definecolor{ansi-green-intense}{HTML}{007427}
    \definecolor{ansi-yellow}{HTML}{DDB62B}
    \definecolor{ansi-yellow-intense}{HTML}{B27D12}
    \definecolor{ansi-blue}{HTML}{208FFB}
    \definecolor{ansi-blue-intense}{HTML}{0065CA}
    \definecolor{ansi-magenta}{HTML}{D160C4}
    \definecolor{ansi-magenta-intense}{HTML}{A03196}
    \definecolor{ansi-cyan}{HTML}{60C6C8}
    \definecolor{ansi-cyan-intense}{HTML}{258F8F}
    \definecolor{ansi-white}{HTML}{C5C1B4}
    \definecolor{ansi-white-intense}{HTML}{A1A6B2}

    % commands and environments needed by pandoc snippets
    % extracted from the output of `pandoc -s`
    \providecommand{\tightlist}{%
      \setlength{\itemsep}{0pt}\setlength{\parskip}{0pt}}
    \DefineVerbatimEnvironment{Highlighting}{Verbatim}{commandchars=\\\{\}}
    % Add ',fontsize=\small' for more characters per line
    \newenvironment{Shaded}{}{}
    \newcommand{\KeywordTok}[1]{\textcolor[rgb]{0.00,0.44,0.13}{\textbf{{#1}}}}
    \newcommand{\DataTypeTok}[1]{\textcolor[rgb]{0.56,0.13,0.00}{{#1}}}
    \newcommand{\DecValTok}[1]{\textcolor[rgb]{0.25,0.63,0.44}{{#1}}}
    \newcommand{\BaseNTok}[1]{\textcolor[rgb]{0.25,0.63,0.44}{{#1}}}
    \newcommand{\FloatTok}[1]{\textcolor[rgb]{0.25,0.63,0.44}{{#1}}}
    \newcommand{\CharTok}[1]{\textcolor[rgb]{0.25,0.44,0.63}{{#1}}}
    \newcommand{\StringTok}[1]{\textcolor[rgb]{0.25,0.44,0.63}{{#1}}}
    \newcommand{\CommentTok}[1]{\textcolor[rgb]{0.38,0.63,0.69}{\textit{{#1}}}}
    \newcommand{\OtherTok}[1]{\textcolor[rgb]{0.00,0.44,0.13}{{#1}}}
    \newcommand{\AlertTok}[1]{\textcolor[rgb]{1.00,0.00,0.00}{\textbf{{#1}}}}
    \newcommand{\FunctionTok}[1]{\textcolor[rgb]{0.02,0.16,0.49}{{#1}}}
    \newcommand{\RegionMarkerTok}[1]{{#1}}
    \newcommand{\ErrorTok}[1]{\textcolor[rgb]{1.00,0.00,0.00}{\textbf{{#1}}}}
    \newcommand{\NormalTok}[1]{{#1}}
    
    % Additional commands for more recent versions of Pandoc
    \newcommand{\ConstantTok}[1]{\textcolor[rgb]{0.53,0.00,0.00}{{#1}}}
    \newcommand{\SpecialCharTok}[1]{\textcolor[rgb]{0.25,0.44,0.63}{{#1}}}
    \newcommand{\VerbatimStringTok}[1]{\textcolor[rgb]{0.25,0.44,0.63}{{#1}}}
    \newcommand{\SpecialStringTok}[1]{\textcolor[rgb]{0.73,0.40,0.53}{{#1}}}
    \newcommand{\ImportTok}[1]{{#1}}
    \newcommand{\DocumentationTok}[1]{\textcolor[rgb]{0.73,0.13,0.13}{\textit{{#1}}}}
    \newcommand{\AnnotationTok}[1]{\textcolor[rgb]{0.38,0.63,0.69}{\textbf{\textit{{#1}}}}}
    \newcommand{\CommentVarTok}[1]{\textcolor[rgb]{0.38,0.63,0.69}{\textbf{\textit{{#1}}}}}
    \newcommand{\VariableTok}[1]{\textcolor[rgb]{0.10,0.09,0.49}{{#1}}}
    \newcommand{\ControlFlowTok}[1]{\textcolor[rgb]{0.00,0.44,0.13}{\textbf{{#1}}}}
    \newcommand{\OperatorTok}[1]{\textcolor[rgb]{0.40,0.40,0.40}{{#1}}}
    \newcommand{\BuiltInTok}[1]{{#1}}
    \newcommand{\ExtensionTok}[1]{{#1}}
    \newcommand{\PreprocessorTok}[1]{\textcolor[rgb]{0.74,0.48,0.00}{{#1}}}
    \newcommand{\AttributeTok}[1]{\textcolor[rgb]{0.49,0.56,0.16}{{#1}}}
    \newcommand{\InformationTok}[1]{\textcolor[rgb]{0.38,0.63,0.69}{\textbf{\textit{{#1}}}}}
    \newcommand{\WarningTok}[1]{\textcolor[rgb]{0.38,0.63,0.69}{\textbf{\textit{{#1}}}}}
    
    
    % Define a nice break command that doesn't care if a line doesn't already
    % exist.
    \def\br{\hspace*{\fill} \\* }
    % Math Jax compatability definitions
    \def\gt{>}
    \def\lt{<}
    % Document parameters
    \title{2.Logistic-Regression}
    
    
    

    % Pygments definitions
    
\makeatletter
\def\PY@reset{\let\PY@it=\relax \let\PY@bf=\relax%
    \let\PY@ul=\relax \let\PY@tc=\relax%
    \let\PY@bc=\relax \let\PY@ff=\relax}
\def\PY@tok#1{\csname PY@tok@#1\endcsname}
\def\PY@toks#1+{\ifx\relax#1\empty\else%
    \PY@tok{#1}\expandafter\PY@toks\fi}
\def\PY@do#1{\PY@bc{\PY@tc{\PY@ul{%
    \PY@it{\PY@bf{\PY@ff{#1}}}}}}}
\def\PY#1#2{\PY@reset\PY@toks#1+\relax+\PY@do{#2}}

\expandafter\def\csname PY@tok@w\endcsname{\def\PY@tc##1{\textcolor[rgb]{0.73,0.73,0.73}{##1}}}
\expandafter\def\csname PY@tok@c\endcsname{\let\PY@it=\textit\def\PY@tc##1{\textcolor[rgb]{0.25,0.50,0.50}{##1}}}
\expandafter\def\csname PY@tok@cp\endcsname{\def\PY@tc##1{\textcolor[rgb]{0.74,0.48,0.00}{##1}}}
\expandafter\def\csname PY@tok@k\endcsname{\let\PY@bf=\textbf\def\PY@tc##1{\textcolor[rgb]{0.00,0.50,0.00}{##1}}}
\expandafter\def\csname PY@tok@kp\endcsname{\def\PY@tc##1{\textcolor[rgb]{0.00,0.50,0.00}{##1}}}
\expandafter\def\csname PY@tok@kt\endcsname{\def\PY@tc##1{\textcolor[rgb]{0.69,0.00,0.25}{##1}}}
\expandafter\def\csname PY@tok@o\endcsname{\def\PY@tc##1{\textcolor[rgb]{0.40,0.40,0.40}{##1}}}
\expandafter\def\csname PY@tok@ow\endcsname{\let\PY@bf=\textbf\def\PY@tc##1{\textcolor[rgb]{0.67,0.13,1.00}{##1}}}
\expandafter\def\csname PY@tok@nb\endcsname{\def\PY@tc##1{\textcolor[rgb]{0.00,0.50,0.00}{##1}}}
\expandafter\def\csname PY@tok@nf\endcsname{\def\PY@tc##1{\textcolor[rgb]{0.00,0.00,1.00}{##1}}}
\expandafter\def\csname PY@tok@nc\endcsname{\let\PY@bf=\textbf\def\PY@tc##1{\textcolor[rgb]{0.00,0.00,1.00}{##1}}}
\expandafter\def\csname PY@tok@nn\endcsname{\let\PY@bf=\textbf\def\PY@tc##1{\textcolor[rgb]{0.00,0.00,1.00}{##1}}}
\expandafter\def\csname PY@tok@ne\endcsname{\let\PY@bf=\textbf\def\PY@tc##1{\textcolor[rgb]{0.82,0.25,0.23}{##1}}}
\expandafter\def\csname PY@tok@nv\endcsname{\def\PY@tc##1{\textcolor[rgb]{0.10,0.09,0.49}{##1}}}
\expandafter\def\csname PY@tok@no\endcsname{\def\PY@tc##1{\textcolor[rgb]{0.53,0.00,0.00}{##1}}}
\expandafter\def\csname PY@tok@nl\endcsname{\def\PY@tc##1{\textcolor[rgb]{0.63,0.63,0.00}{##1}}}
\expandafter\def\csname PY@tok@ni\endcsname{\let\PY@bf=\textbf\def\PY@tc##1{\textcolor[rgb]{0.60,0.60,0.60}{##1}}}
\expandafter\def\csname PY@tok@na\endcsname{\def\PY@tc##1{\textcolor[rgb]{0.49,0.56,0.16}{##1}}}
\expandafter\def\csname PY@tok@nt\endcsname{\let\PY@bf=\textbf\def\PY@tc##1{\textcolor[rgb]{0.00,0.50,0.00}{##1}}}
\expandafter\def\csname PY@tok@nd\endcsname{\def\PY@tc##1{\textcolor[rgb]{0.67,0.13,1.00}{##1}}}
\expandafter\def\csname PY@tok@s\endcsname{\def\PY@tc##1{\textcolor[rgb]{0.73,0.13,0.13}{##1}}}
\expandafter\def\csname PY@tok@sd\endcsname{\let\PY@it=\textit\def\PY@tc##1{\textcolor[rgb]{0.73,0.13,0.13}{##1}}}
\expandafter\def\csname PY@tok@si\endcsname{\let\PY@bf=\textbf\def\PY@tc##1{\textcolor[rgb]{0.73,0.40,0.53}{##1}}}
\expandafter\def\csname PY@tok@se\endcsname{\let\PY@bf=\textbf\def\PY@tc##1{\textcolor[rgb]{0.73,0.40,0.13}{##1}}}
\expandafter\def\csname PY@tok@sr\endcsname{\def\PY@tc##1{\textcolor[rgb]{0.73,0.40,0.53}{##1}}}
\expandafter\def\csname PY@tok@ss\endcsname{\def\PY@tc##1{\textcolor[rgb]{0.10,0.09,0.49}{##1}}}
\expandafter\def\csname PY@tok@sx\endcsname{\def\PY@tc##1{\textcolor[rgb]{0.00,0.50,0.00}{##1}}}
\expandafter\def\csname PY@tok@m\endcsname{\def\PY@tc##1{\textcolor[rgb]{0.40,0.40,0.40}{##1}}}
\expandafter\def\csname PY@tok@gh\endcsname{\let\PY@bf=\textbf\def\PY@tc##1{\textcolor[rgb]{0.00,0.00,0.50}{##1}}}
\expandafter\def\csname PY@tok@gu\endcsname{\let\PY@bf=\textbf\def\PY@tc##1{\textcolor[rgb]{0.50,0.00,0.50}{##1}}}
\expandafter\def\csname PY@tok@gd\endcsname{\def\PY@tc##1{\textcolor[rgb]{0.63,0.00,0.00}{##1}}}
\expandafter\def\csname PY@tok@gi\endcsname{\def\PY@tc##1{\textcolor[rgb]{0.00,0.63,0.00}{##1}}}
\expandafter\def\csname PY@tok@gr\endcsname{\def\PY@tc##1{\textcolor[rgb]{1.00,0.00,0.00}{##1}}}
\expandafter\def\csname PY@tok@ge\endcsname{\let\PY@it=\textit}
\expandafter\def\csname PY@tok@gs\endcsname{\let\PY@bf=\textbf}
\expandafter\def\csname PY@tok@gp\endcsname{\let\PY@bf=\textbf\def\PY@tc##1{\textcolor[rgb]{0.00,0.00,0.50}{##1}}}
\expandafter\def\csname PY@tok@go\endcsname{\def\PY@tc##1{\textcolor[rgb]{0.53,0.53,0.53}{##1}}}
\expandafter\def\csname PY@tok@gt\endcsname{\def\PY@tc##1{\textcolor[rgb]{0.00,0.27,0.87}{##1}}}
\expandafter\def\csname PY@tok@err\endcsname{\def\PY@bc##1{\setlength{\fboxsep}{0pt}\fcolorbox[rgb]{1.00,0.00,0.00}{1,1,1}{\strut ##1}}}
\expandafter\def\csname PY@tok@kc\endcsname{\let\PY@bf=\textbf\def\PY@tc##1{\textcolor[rgb]{0.00,0.50,0.00}{##1}}}
\expandafter\def\csname PY@tok@kd\endcsname{\let\PY@bf=\textbf\def\PY@tc##1{\textcolor[rgb]{0.00,0.50,0.00}{##1}}}
\expandafter\def\csname PY@tok@kn\endcsname{\let\PY@bf=\textbf\def\PY@tc##1{\textcolor[rgb]{0.00,0.50,0.00}{##1}}}
\expandafter\def\csname PY@tok@kr\endcsname{\let\PY@bf=\textbf\def\PY@tc##1{\textcolor[rgb]{0.00,0.50,0.00}{##1}}}
\expandafter\def\csname PY@tok@bp\endcsname{\def\PY@tc##1{\textcolor[rgb]{0.00,0.50,0.00}{##1}}}
\expandafter\def\csname PY@tok@fm\endcsname{\def\PY@tc##1{\textcolor[rgb]{0.00,0.00,1.00}{##1}}}
\expandafter\def\csname PY@tok@vc\endcsname{\def\PY@tc##1{\textcolor[rgb]{0.10,0.09,0.49}{##1}}}
\expandafter\def\csname PY@tok@vg\endcsname{\def\PY@tc##1{\textcolor[rgb]{0.10,0.09,0.49}{##1}}}
\expandafter\def\csname PY@tok@vi\endcsname{\def\PY@tc##1{\textcolor[rgb]{0.10,0.09,0.49}{##1}}}
\expandafter\def\csname PY@tok@vm\endcsname{\def\PY@tc##1{\textcolor[rgb]{0.10,0.09,0.49}{##1}}}
\expandafter\def\csname PY@tok@sa\endcsname{\def\PY@tc##1{\textcolor[rgb]{0.73,0.13,0.13}{##1}}}
\expandafter\def\csname PY@tok@sb\endcsname{\def\PY@tc##1{\textcolor[rgb]{0.73,0.13,0.13}{##1}}}
\expandafter\def\csname PY@tok@sc\endcsname{\def\PY@tc##1{\textcolor[rgb]{0.73,0.13,0.13}{##1}}}
\expandafter\def\csname PY@tok@dl\endcsname{\def\PY@tc##1{\textcolor[rgb]{0.73,0.13,0.13}{##1}}}
\expandafter\def\csname PY@tok@s2\endcsname{\def\PY@tc##1{\textcolor[rgb]{0.73,0.13,0.13}{##1}}}
\expandafter\def\csname PY@tok@sh\endcsname{\def\PY@tc##1{\textcolor[rgb]{0.73,0.13,0.13}{##1}}}
\expandafter\def\csname PY@tok@s1\endcsname{\def\PY@tc##1{\textcolor[rgb]{0.73,0.13,0.13}{##1}}}
\expandafter\def\csname PY@tok@mb\endcsname{\def\PY@tc##1{\textcolor[rgb]{0.40,0.40,0.40}{##1}}}
\expandafter\def\csname PY@tok@mf\endcsname{\def\PY@tc##1{\textcolor[rgb]{0.40,0.40,0.40}{##1}}}
\expandafter\def\csname PY@tok@mh\endcsname{\def\PY@tc##1{\textcolor[rgb]{0.40,0.40,0.40}{##1}}}
\expandafter\def\csname PY@tok@mi\endcsname{\def\PY@tc##1{\textcolor[rgb]{0.40,0.40,0.40}{##1}}}
\expandafter\def\csname PY@tok@il\endcsname{\def\PY@tc##1{\textcolor[rgb]{0.40,0.40,0.40}{##1}}}
\expandafter\def\csname PY@tok@mo\endcsname{\def\PY@tc##1{\textcolor[rgb]{0.40,0.40,0.40}{##1}}}
\expandafter\def\csname PY@tok@ch\endcsname{\let\PY@it=\textit\def\PY@tc##1{\textcolor[rgb]{0.25,0.50,0.50}{##1}}}
\expandafter\def\csname PY@tok@cm\endcsname{\let\PY@it=\textit\def\PY@tc##1{\textcolor[rgb]{0.25,0.50,0.50}{##1}}}
\expandafter\def\csname PY@tok@cpf\endcsname{\let\PY@it=\textit\def\PY@tc##1{\textcolor[rgb]{0.25,0.50,0.50}{##1}}}
\expandafter\def\csname PY@tok@c1\endcsname{\let\PY@it=\textit\def\PY@tc##1{\textcolor[rgb]{0.25,0.50,0.50}{##1}}}
\expandafter\def\csname PY@tok@cs\endcsname{\let\PY@it=\textit\def\PY@tc##1{\textcolor[rgb]{0.25,0.50,0.50}{##1}}}

\def\PYZbs{\char`\\}
\def\PYZus{\char`\_}
\def\PYZob{\char`\{}
\def\PYZcb{\char`\}}
\def\PYZca{\char`\^}
\def\PYZam{\char`\&}
\def\PYZlt{\char`\<}
\def\PYZgt{\char`\>}
\def\PYZsh{\char`\#}
\def\PYZpc{\char`\%}
\def\PYZdl{\char`\$}
\def\PYZhy{\char`\-}
\def\PYZsq{\char`\'}
\def\PYZdq{\char`\"}
\def\PYZti{\char`\~}
% for compatibility with earlier versions
\def\PYZat{@}
\def\PYZlb{[}
\def\PYZrb{]}
\makeatother


    % Exact colors from NB
    \definecolor{incolor}{rgb}{0.0, 0.0, 0.5}
    \definecolor{outcolor}{rgb}{0.545, 0.0, 0.0}



    
    % Prevent overflowing lines due to hard-to-break entities
    \sloppy 
    % Setup hyperref package
    \hypersetup{
      breaklinks=true,  % so long urls are correctly broken across lines
      colorlinks=true,
      urlcolor=urlcolor,
      linkcolor=linkcolor,
      citecolor=citecolor,
      }
    % Slightly bigger margins than the latex defaults
    
    \geometry{verbose,tmargin=1in,bmargin=1in,lmargin=1in,rmargin=1in}
    
    

    \begin{document}
    
    
    \maketitle
    
    

    
    \subsection{Logistic Regression using
TensorFlow}\label{logistic-regression-using-tensorflow}

\begin{itemize}
\item
  This notebook is adapted from
  \href{https://github.com/aymericdamien/TensorFlow-Examples/blob/master/notebooks/2_BasicModels/logistic_regression.ipynb}{Aymeric
  Damian's logistic regression notebook}
\item
  Clone the full collection
  \href{https://github.com/aymericdamien/TensorFlow-Examples}{here}.
\end{itemize}

    \subsubsection{multi-label logistic
regression}\label{multi-label-logistic-regression}

Is the next step after linear regression. * Like linear regression,
there are only an input layer and and output layer. * Unlike linear
regression, the relation of output to input is not linear. * Also, we
have ten output nodes, instead of one.

    \href{https://en.wikipedia.org/wiki/Logistic_regression}{Logistic
regression} refers to a soft-classifier over \(k\) classes based on a
linear function of the input. We will look at an example where we want
to classify handwritten digits into one of \(k=10\) classes: \(0-9\)

The logistic regression model works in a similar fashion to a linear
regression model except that the final sum of the product between the
weights and dependent variable is passed through a function that
transforms the unbounded outputs of the linear operation into a
normalized conditional probability over the \(k\) classes.

    \subsection{MNIST Dataset Overview}\label{mnist-dataset-overview}

This example is using MNIST handwritten digits. The dataset contains
60,000 examples for training and 10,000 examples for testing. The digits
have been size-normalized and centered in a fixed-size image (28x28
pixels) with values from 0 to 1. For simplicity, each image has been
flattened and converted to a 1-D numpy array of 784 features (28*28).

\begin{figure}
\centering
\includegraphics{http://neuralnetworksanddeeplearning.com/images/mnist_100_digits.png}
\caption{MNIST Dataset}
\end{figure}

More info: http://yann.lecun.com/exdb/mnist/

    \subsection{The logistic model}\label{the-logistic-model}

We use the logistic model as a classifier which maps each digit image to
an integer number between 0 and 9 which corresponds to the identity of
the digit

    The inputs (placeholders) are:

\begin{itemize}
\tightlist
\item
  \(X\) - a 784 dimensional vector.
\end{itemize}

    \begin{itemize}
\tightlist
\item
  \(y\) - the label corresponding to \(X\). Encoded using 1-hot
  encoding. I.e. \textbf{0}=(1,0,...0), \textbf{1}=(0,1,0,....,0) etc.
\end{itemize}

    There are 10 sets of parameters, one for each digit \(j=0,\ldots,9\) : *
\(W_j\)= a 784 dimensional vector * \(b_j\)= a scalar.

    The logistic funcion defines a distribution over the digits. We predict
with the digit with the highest probability. \[
p(y=j | X) = g(s_j)\;\;\mbox{ where }\;\; s_j=W_j \cdot X +b_j \]

    \[\mbox{and  } g(s_j) = \frac{\exp(s_j)}{\sum_{i=0}^9 \exp(s_i)}
\]is the \href{https://en.wikipedia.org/wiki/softmax_function}{softmax
function}

    \subsubsection{The cross-entropy cost}\label{the-cross-entropy-cost}

As our model outputs a vector of 10 conditional probabilities we use the
negative cross-entropy as the cost function:
\[ Cost \left(\{W_j,b_j\}_{j=0}^9\right)
=-\frac{1}{N} \sum_{i=1}^N \sum_{j=0}^n y^i_j \log g(s^i_j) \]

    \subsubsection{Data Flow Diagram}\label{data-flow-diagram}

    \subsection{Coding Logistic Regression in
Tensorflow}\label{coding-logistic-regression-in-tensorflow}

    \begin{Verbatim}[commandchars=\\\{\}]
{\color{incolor}In [{\color{incolor}1}]:} \PY{k+kn}{import} \PY{n+nn}{tensorflow} \PY{k}{as} \PY{n+nn}{tf}
        \PY{k+kn}{import} \PY{n+nn}{numpy} \PY{k}{as} \PY{n+nn}{np}
        \PY{k+kn}{import} \PY{n+nn}{matplotlib}\PY{n+nn}{.}\PY{n+nn}{pyplot} \PY{k}{as} \PY{n+nn}{plt} 
        
        \PY{k+kn}{import} \PY{n+nn}{warnings}
        \PY{c+c1}{\PYZsh{}\PYZsh{} Tensorflow produces a lot of warnings. We generally want to suppress them. The below code does exactly that. }
        \PY{n}{warnings}\PY{o}{.}\PY{n}{filterwarnings}\PY{p}{(}\PY{l+s+s1}{\PYZsq{}}\PY{l+s+s1}{ignore}\PY{l+s+s1}{\PYZsq{}}\PY{p}{)}
        \PY{n}{tf}\PY{o}{.}\PY{n}{logging}\PY{o}{.}\PY{n}{set\PYZus{}verbosity}\PY{p}{(}\PY{n}{tf}\PY{o}{.}\PY{n}{logging}\PY{o}{.}\PY{n}{ERROR}\PY{p}{)}
        
        \PY{n}{rng} \PY{o}{=} \PY{n}{np}\PY{o}{.}\PY{n}{random}
        \PY{n}{logs\PYZus{}path} \PY{o}{=} \PY{l+s+s1}{\PYZsq{}}\PY{l+s+s1}{logs/Logistic\PYZus{}regression}\PY{l+s+s1}{\PYZsq{}}
\end{Verbatim}


    \begin{Verbatim}[commandchars=\\\{\}]
{\color{incolor}In [{\color{incolor}2}]:} \PY{c+c1}{\PYZsh{} Import MNIST data}
        \PY{k+kn}{from} \PY{n+nn}{tensorflow}\PY{n+nn}{.}\PY{n+nn}{examples}\PY{n+nn}{.}\PY{n+nn}{tutorials}\PY{n+nn}{.}\PY{n+nn}{mnist} \PY{k}{import} \PY{n}{input\PYZus{}data}
        \PY{n}{mnist} \PY{o}{=} \PY{n}{input\PYZus{}data}\PY{o}{.}\PY{n}{read\PYZus{}data\PYZus{}sets}\PY{p}{(}\PY{l+s+s2}{\PYZdq{}}\PY{l+s+s2}{/tmp/data/}\PY{l+s+s2}{\PYZdq{}}\PY{p}{,} \PY{n}{one\PYZus{}hot}\PY{o}{=}\PY{k+kc}{True}\PY{p}{)}
\end{Verbatim}


    \begin{Verbatim}[commandchars=\\\{\}]
Successfully downloaded train-images-idx3-ubyte.gz 9912422 bytes.
Extracting /tmp/data/train-images-idx3-ubyte.gz
Successfully downloaded train-labels-idx1-ubyte.gz 28881 bytes.
Extracting /tmp/data/train-labels-idx1-ubyte.gz
Successfully downloaded t10k-images-idx3-ubyte.gz 1648877 bytes.
Extracting /tmp/data/t10k-images-idx3-ubyte.gz
Successfully downloaded t10k-labels-idx1-ubyte.gz 4542 bytes.
Extracting /tmp/data/t10k-labels-idx1-ubyte.gz

    \end{Verbatim}

    \subsubsection{Defining the logistic
model}\label{defining-the-logistic-model}

    \begin{Shaded}
\begin{Highlighting}[]
\CommentTok{# Placeholder1: flattened images of dimension 28*28 = 784}
\NormalTok{x }\OperatorTok{=}\NormalTok{ tf.placeholder(dtype }\OperatorTok{=}\NormalTok{ tf.float32, shape }\OperatorTok{=}\NormalTok{ [}\VariableTok{None}\NormalTok{, }\DecValTok{784}\NormalTok{], name }\OperatorTok{=} \StringTok{"inputData"}\NormalTok{) }
\CommentTok{# Placeholder2: one-hot encoded labels for the 10 classes}
\NormalTok{y }\OperatorTok{=}\NormalTok{ tf.placeholder(dtype }\OperatorTok{=}\NormalTok{ tf.float32, shape }\OperatorTok{=}\NormalTok{ [}\VariableTok{None}\NormalTok{, }\DecValTok{10}\NormalTok{], name }\OperatorTok{=} \StringTok{"actualLabel"}\NormalTok{)}

\NormalTok{W }\OperatorTok{=}\NormalTok{ tf.Variable(initial_value }\OperatorTok{=}\NormalTok{ tf.zeros([}\DecValTok{784}\NormalTok{, }\DecValTok{10}\NormalTok{]), name }\OperatorTok{=} \StringTok{"weight"}\NormalTok{)}
\NormalTok{b }\OperatorTok{=}\NormalTok{ tf.Variable(initial_value }\OperatorTok{=}\NormalTok{ tf.zeros([}\DecValTok{10}\NormalTok{]), name }\OperatorTok{=} \StringTok{"bias"}\NormalTok{)}

\ControlFlowTok{with}\NormalTok{ tf.name_scope(}\StringTok{'model'}\NormalTok{):}
\NormalTok{    prediction }\OperatorTok{=}\NormalTok{ tf.nn.softmax(tf.add(b, tf.matmul(x, W))) }
\end{Highlighting}
\end{Shaded}

    \begin{itemize}
\item
  The operation defined in out model is:

\begin{Shaded}
\begin{Highlighting}[]
\NormalTok{prediction }\OperatorTok{=}\NormalTok{ tf.nn.softmax(tf.add(b, tf.matmul(x, W))) }\CommentTok{# Softmax}
\end{Highlighting}
\end{Shaded}
\item
  it is a composition of:
\item
  \texttt{tf.matmul(x,\ W))} : performs dot product between the input
  vector \texttt{x} and the weights matrix \texttt{W}, yielding a vector
  of dimension 10.
\item
  \texttt{tf.add(b,\ tf.matmul(x,W))} : returns the tensor sum between
  the tensors b and the output of the inner computation
\item
  \texttt{tf.nn.softmax(A)} : applies the softmax function on each value
  of the input tensor (default is along the first dimension)
\end{itemize}

    \begin{Verbatim}[commandchars=\\\{\}]
{\color{incolor}In [{\color{incolor}3}]:} \PY{c+c1}{\PYZsh{} Lets run the code we just described}
        
        \PY{n}{x} \PY{o}{=} \PY{n}{tf}\PY{o}{.}\PY{n}{placeholder}\PY{p}{(}\PY{n}{dtype} \PY{o}{=} \PY{n}{tf}\PY{o}{.}\PY{n}{float32}\PY{p}{,} \PY{n}{shape} \PY{o}{=} \PY{p}{[}\PY{k+kc}{None}\PY{p}{,} \PY{l+m+mi}{784}\PY{p}{]}\PY{p}{,} \PY{n}{name} \PY{o}{=} \PY{l+s+s2}{\PYZdq{}}\PY{l+s+s2}{inputFeatures}\PY{l+s+s2}{\PYZdq{}}\PY{p}{)} \PY{c+c1}{\PYZsh{} mnist data image of shape 28*28=784}
        \PY{n}{y} \PY{o}{=} \PY{n}{tf}\PY{o}{.}\PY{n}{placeholder}\PY{p}{(}\PY{n}{dtype} \PY{o}{=} \PY{n}{tf}\PY{o}{.}\PY{n}{float32}\PY{p}{,} \PY{n}{shape} \PY{o}{=} \PY{p}{[}\PY{k+kc}{None}\PY{p}{,} \PY{l+m+mi}{10}\PY{p}{]}\PY{p}{,} \PY{n}{name} \PY{o}{=} \PY{l+s+s2}{\PYZdq{}}\PY{l+s+s2}{actualLabel}\PY{l+s+s2}{\PYZdq{}}\PY{p}{)} \PY{c+c1}{\PYZsh{} 0\PYZhy{}9 digits recognition =\PYZgt{} 10 classes}
        
        \PY{n}{W} \PY{o}{=} \PY{n}{tf}\PY{o}{.}\PY{n}{Variable}\PY{p}{(}\PY{n}{initial\PYZus{}value} \PY{o}{=} \PY{n}{tf}\PY{o}{.}\PY{n}{zeros}\PY{p}{(}\PY{p}{[}\PY{l+m+mi}{784}\PY{p}{,} \PY{l+m+mi}{10}\PY{p}{]}\PY{p}{)}\PY{p}{,} \PY{n}{name} \PY{o}{=} \PY{l+s+s2}{\PYZdq{}}\PY{l+s+s2}{weight}\PY{l+s+s2}{\PYZdq{}}\PY{p}{)}
        \PY{n}{b} \PY{o}{=} \PY{n}{tf}\PY{o}{.}\PY{n}{Variable}\PY{p}{(}\PY{n}{initial\PYZus{}value} \PY{o}{=} \PY{n}{tf}\PY{o}{.}\PY{n}{zeros}\PY{p}{(}\PY{p}{[}\PY{l+m+mi}{10}\PY{p}{]}\PY{p}{)}\PY{p}{,} \PY{n}{name} \PY{o}{=} \PY{l+s+s2}{\PYZdq{}}\PY{l+s+s2}{bias}\PY{l+s+s2}{\PYZdq{}}\PY{p}{)}
        
        \PY{k}{with} \PY{n}{tf}\PY{o}{.}\PY{n}{name\PYZus{}scope}\PY{p}{(}\PY{l+s+s1}{\PYZsq{}}\PY{l+s+s1}{model}\PY{l+s+s1}{\PYZsq{}}\PY{p}{)}\PY{p}{:}
            \PY{n}{prediction} \PY{o}{=} \PY{n}{tf}\PY{o}{.}\PY{n}{nn}\PY{o}{.}\PY{n}{softmax}\PY{p}{(}\PY{n}{tf}\PY{o}{.}\PY{n}{add}\PY{p}{(}\PY{n}{b}\PY{p}{,} \PY{n}{tf}\PY{o}{.}\PY{n}{matmul}\PY{p}{(}\PY{n}{x}\PY{p}{,} \PY{n}{W}\PY{p}{)}\PY{p}{)}\PY{p}{)} \PY{c+c1}{\PYZsh{} Softmax}
\end{Verbatim}


    \subsubsection{Adding a regularization
term}\label{adding-a-regularization-term}

One way to reduce over-fitting is to add a \textbf{regularization term}
to the loss. This term is also referred to as \textbf{weight decay}
because it pushes the weights towards zero.

    We use an L2 regularizer, given the weight vectors \(W_j\) and the
biases \(b_j\) the regularization term is
\[ l2\left(\{W_j,b_j\}_{j=0}^9\right) = \sum_{j=0}^9  \left[\sum_{i=1}^{784} W_{ji}^2 +b_j^2\right] \]

    \begin{Verbatim}[commandchars=\\\{\}]
{\color{incolor}In [{\color{incolor}4}]:} \PY{c+c1}{\PYZsh{} Parameters}
        \PY{c+c1}{\PYZsh{}learning\PYZus{}rate = 0.01}
        \PY{n}{training\PYZus{}epochs} \PY{o}{=} \PY{l+m+mi}{25}
        \PY{n}{batch\PYZus{}size} \PY{o}{=} \PY{l+m+mi}{100}
        \PY{n}{display\PYZus{}step} \PY{o}{=} \PY{l+m+mi}{5}
        \PY{n}{lamb} \PY{o}{=} \PY{l+m+mf}{0.01} \PY{c+c1}{\PYZsh{}This is the hyperparameter that controls the strength of the regularization}
        
        \PY{c+c1}{\PYZsh{} Minimize error using cross entropy loss}
        \PY{c+c1}{\PYZsh{} reduce\PYZus{}mean calculates the mean across dimensions of a tensor}
        \PY{n}{loss} \PY{o}{=} \PY{n}{tf}\PY{o}{.}\PY{n}{reduce\PYZus{}mean}\PY{p}{(}\PY{o}{\PYZhy{}}\PY{n}{tf}\PY{o}{.}\PY{n}{reduce\PYZus{}sum}\PY{p}{(}\PY{n}{y}\PY{o}{*}\PY{n}{tf}\PY{o}{.}\PY{n}{log}\PY{p}{(}\PY{n}{prediction}\PY{p}{)}\PY{p}{,} \PY{n}{axis}\PY{o}{=}\PY{l+m+mi}{1}\PY{p}{)}  
                              \PY{o}{+} \PY{n}{lamb} \PY{o}{*} \PY{p}{(}\PY{n}{tf}\PY{o}{.}\PY{n}{nn}\PY{o}{.}\PY{n}{l2\PYZus{}loss}\PY{p}{(}\PY{n}{W}\PY{p}{)} \PY{o}{+} \PY{n}{tf}\PY{o}{.}\PY{n}{nn}\PY{o}{.}\PY{n}{l2\PYZus{}loss}\PY{p}{(}\PY{n}{b}\PY{p}{)}\PY{p}{)}\PY{p}{)}
                             
        \PY{c+c1}{\PYZsh{} Logging commands}
        \PY{n}{tf}\PY{o}{.}\PY{n}{summary}\PY{o}{.}\PY{n}{scalar}\PY{p}{(}\PY{l+s+s2}{\PYZdq{}}\PY{l+s+s2}{loss}\PY{l+s+s2}{\PYZdq{}}\PY{p}{,} \PY{n}{loss}\PY{p}{)}
        \PY{n}{merged\PYZus{}summary\PYZus{}op} \PY{o}{=} \PY{n}{tf}\PY{o}{.}\PY{n}{summary}\PY{o}{.}\PY{n}{merge\PYZus{}all}\PY{p}{(}\PY{p}{)}
                                  
        \PY{k}{with} \PY{n}{tf}\PY{o}{.}\PY{n}{name\PYZus{}scope}\PY{p}{(}\PY{l+s+s1}{\PYZsq{}}\PY{l+s+s1}{Optimizer}\PY{l+s+s1}{\PYZsq{}}\PY{p}{)}\PY{p}{:}
            \PY{n}{optimizer} \PY{o}{=} \PY{n}{tf}\PY{o}{.}\PY{n}{train}\PY{o}{.}\PY{n}{GradientDescentOptimizer}\PY{p}{(}\PY{l+m+mf}{0.1}\PY{p}{)}\PY{o}{.}\PY{n}{minimize}\PY{p}{(}\PY{n}{loss}\PY{p}{)}
\end{Verbatim}


    \begin{Verbatim}[commandchars=\\\{\}]
{\color{incolor}In [{\color{incolor}5}]:} \PY{n}{init} \PY{o}{=} \PY{n}{tf}\PY{o}{.}\PY{n}{global\PYZus{}variables\PYZus{}initializer}\PY{p}{(}\PY{p}{)}
\end{Verbatim}


    \subsubsection{Executing the optimization in a
session}\label{executing-the-optimization-in-a-session}

    \begin{Verbatim}[commandchars=\\\{\}]
{\color{incolor}In [{\color{incolor}6}]:} \PY{c+c1}{\PYZsh{} Start training}
        \PY{n}{sess}\PY{o}{=}\PY{n}{tf}\PY{o}{.}\PY{n}{Session}\PY{p}{(}\PY{p}{)}
        \PY{n}{sess}\PY{o}{.}\PY{n}{run}\PY{p}{(}\PY{n}{init}\PY{p}{)}
        
        \PY{n}{summary\PYZus{}writer} \PY{o}{=} \PY{n}{tf}\PY{o}{.}\PY{n}{summary}\PY{o}{.}\PY{n}{FileWriter}\PY{p}{(}\PY{n}{logs\PYZus{}path} \PY{o}{+} \PY{l+s+s2}{\PYZdq{}}\PY{l+s+s2}{/logistic}\PY{l+s+s2}{\PYZdq{}}\PY{p}{,} \PY{n}{graph}\PY{o}{=}\PY{n}{tf}\PY{o}{.}\PY{n}{get\PYZus{}default\PYZus{}graph}\PY{p}{(}\PY{p}{)}\PY{p}{)}
\end{Verbatim}


    \begin{Verbatim}[commandchars=\\\{\}]
{\color{incolor}In [{\color{incolor}7}]:} \PY{c+c1}{\PYZsh{} Training cycle}
        \PY{k}{for} \PY{n}{epoch} \PY{o+ow}{in} \PY{n+nb}{range}\PY{p}{(}\PY{n}{training\PYZus{}epochs}\PY{p}{)}\PY{p}{:}
            \PY{n}{avg\PYZus{}loss} \PY{o}{=} \PY{l+m+mf}{0.}
            \PY{n}{total\PYZus{}batch} \PY{o}{=} \PY{n+nb}{int}\PY{p}{(}\PY{n}{mnist}\PY{o}{.}\PY{n}{train}\PY{o}{.}\PY{n}{num\PYZus{}examples}\PY{o}{/}\PY{n}{batch\PYZus{}size}\PY{p}{)} \PY{c+c1}{\PYZsh{} there would be 600 batches}
        
            \PY{c+c1}{\PYZsh{} Loop over all batches}
            \PY{k}{for} \PY{n}{i} \PY{o+ow}{in} \PY{n+nb}{range}\PY{p}{(}\PY{n}{total\PYZus{}batch}\PY{p}{)}\PY{p}{:}
                \PY{n}{batch\PYZus{}xs}\PY{p}{,} \PY{n}{batch\PYZus{}ys} \PY{o}{=} \PY{n}{mnist}\PY{o}{.}\PY{n}{train}\PY{o}{.}\PY{n}{next\PYZus{}batch}\PY{p}{(}\PY{n}{batch\PYZus{}size}\PY{p}{)}
        
                \PY{c+c1}{\PYZsh{} Fit training using batch data}
                \PY{n}{\PYZus{}}\PY{p}{,} \PY{n}{c} \PY{o}{=} \PY{n}{sess}\PY{o}{.}\PY{n}{run}\PY{p}{(}\PY{p}{[}\PY{n}{optimizer}\PY{p}{,} \PY{n}{loss}\PY{p}{]}\PY{p}{,} \PY{n}{feed\PYZus{}dict}\PY{o}{=}\PY{p}{\PYZob{}}\PY{n}{x}\PY{p}{:} \PY{n}{batch\PYZus{}xs}\PY{p}{,}
                                                              \PY{n}{y}\PY{p}{:} \PY{n}{batch\PYZus{}ys}\PY{p}{\PYZcb{}}\PY{p}{)}
                \PY{c+c1}{\PYZsh{} Compute average loss}
                \PY{n}{avg\PYZus{}loss} \PY{o}{+}\PY{o}{=} \PY{n}{c} \PY{o}{/} \PY{n}{total\PYZus{}batch}
        
            \PY{c+c1}{\PYZsh{} Display logs per epoch step}
            \PY{k}{if} \PY{p}{(}\PY{n}{epoch}\PY{o}{+}\PY{l+m+mi}{1}\PY{p}{)} \PY{o}{\PYZpc{}} \PY{n}{display\PYZus{}step} \PY{o}{==} \PY{l+m+mi}{0}\PY{p}{:}
                \PY{n+nb}{print}\PY{p}{(}\PY{l+s+s2}{\PYZdq{}}\PY{l+s+s2}{Epoch:}\PY{l+s+s2}{\PYZdq{}}\PY{p}{,} \PY{l+s+s1}{\PYZsq{}}\PY{l+s+si}{\PYZpc{}04d}\PY{l+s+s1}{\PYZsq{}} \PY{o}{\PYZpc{}} \PY{p}{(}\PY{n}{epoch}\PY{o}{+}\PY{l+m+mi}{1}\PY{p}{)}\PY{p}{,} \PY{l+s+s2}{\PYZdq{}}\PY{l+s+s2}{loss=}\PY{l+s+s2}{\PYZdq{}}\PY{p}{,} \PY{l+s+s2}{\PYZdq{}}\PY{l+s+si}{\PYZob{}:.9f\PYZcb{}}\PY{l+s+s2}{\PYZdq{}}\PY{o}{.}\PY{n}{format}\PY{p}{(}\PY{n}{avg\PYZus{}loss}\PY{p}{)}\PY{p}{)}
        
        \PY{n+nb}{print}\PY{p}{(}\PY{l+s+s2}{\PYZdq{}}\PY{l+s+s2}{Optimization Finished!}\PY{l+s+s2}{\PYZdq{}}\PY{p}{)}
\end{Verbatim}


    \begin{Verbatim}[commandchars=\\\{\}]
Epoch: 0005 loss= 0.562108982
Epoch: 0010 loss= 0.562087742
Epoch: 0015 loss= 0.561949274
Epoch: 0020 loss= 0.561928227
Epoch: 0025 loss= 0.561720640
Optimization Finished!

    \end{Verbatim}

    \paragraph{Optional excercise}\label{optional-excercise}

\begin{itemize}
\tightlist
\item
  Add to the log print line the two components of the loss: the entropy
  loss and the regularization term.
\end{itemize}

    \begin{Verbatim}[commandchars=\\\{\}]
{\color{incolor}In [{\color{incolor}8}]:} \PY{c+c1}{\PYZsh{} Calculate test set accuracy, i.e. number of mistakes final model makes on test set}
        \PY{n}{correct\PYZus{}prediction} \PY{o}{=} \PY{n}{tf}\PY{o}{.}\PY{n}{equal}\PY{p}{(}\PY{n}{tf}\PY{o}{.}\PY{n}{argmax}\PY{p}{(}\PY{n}{prediction}\PY{p}{,} \PY{l+m+mi}{1}\PY{p}{)}\PY{p}{,} \PY{n}{tf}\PY{o}{.}\PY{n}{argmax}\PY{p}{(}\PY{n}{y}\PY{p}{,} \PY{l+m+mi}{1}\PY{p}{)}\PY{p}{)}
        
        \PY{c+c1}{\PYZsh{} Calculate accuracy for 3000 examples}
        \PY{n}{accuracy} \PY{o}{=} \PY{n}{tf}\PY{o}{.}\PY{n}{reduce\PYZus{}mean}\PY{p}{(}\PY{n}{tf}\PY{o}{.}\PY{n}{cast}\PY{p}{(}\PY{n}{correct\PYZus{}prediction}\PY{p}{,} \PY{n}{tf}\PY{o}{.}\PY{n}{float32}\PY{p}{)}\PY{p}{)}
        \PY{n+nb}{print}\PY{p}{(}\PY{l+s+s2}{\PYZdq{}}\PY{l+s+s2}{Accuracy:}\PY{l+s+s2}{\PYZdq{}}\PY{p}{,} \PY{n}{accuracy}\PY{o}{.}\PY{n}{eval}\PY{p}{(}\PY{p}{\PYZob{}}\PY{n}{x}\PY{p}{:} \PY{n}{mnist}\PY{o}{.}\PY{n}{test}\PY{o}{.}\PY{n}{images}\PY{p}{[}\PY{p}{:}\PY{l+m+mi}{3000}\PY{p}{]}\PY{p}{,} \PY{n}{y}\PY{p}{:} \PY{n}{mnist}\PY{o}{.}\PY{n}{test}\PY{o}{.}\PY{n}{labels}\PY{p}{[}\PY{p}{:}\PY{l+m+mi}{3000}\PY{p}{]}\PY{p}{\PYZcb{}}\PY{p}{,}\PY{n}{session}\PY{o}{=}\PY{n}{sess}\PY{p}{)}\PY{p}{)}
\end{Verbatim}


    \begin{Verbatim}[commandchars=\\\{\}]
Accuracy: 0.879

    \end{Verbatim}

    \subsection{Using Tensorboard to View Graph
Structure}\label{using-tensorboard-to-view-graph-structure}

    We can have a look at the computational graph that we have just defined
on Tensorboard. We have installed a jupyter extension that makes
connecting to Tensorboard very simple. To do this,

In your Jupyter directory tree view, select the log directory for lesson
1 and click the \textbf{Tensorboard} button as shown in the picture. 

    Next, go to the \textbf{Running} tab, and choose the Tensorboard
instance corresponding to the correct log directory as shown in the
screenshot. 


    % Add a bibliography block to the postdoc
    
    
    
    \end{document}
